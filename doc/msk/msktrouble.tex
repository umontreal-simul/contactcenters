\section{Troubleshooting}
\label{sec:msktrouble}

Many kinds of problems can occur while running the simulator.
Some installation or configuration problems might prevent the
execution of the program.
Syntax or grammatical errors in the XML parameter files also abort the
execution of the simulator.
Even valid XML files may lead to execution errors.
Moreover, errors can happen during simulation, prevents results from
being displayed.
In this section, we examine all these types of errors, and propose
some solutions.
However, this does not cover all the possible errors that can occur.

\subsection{Commands not found or
  \texttt{No\-Class\-Def\-Found\-Error} messages}

An installation problem occurs if a command or class cannot be found.
In the first case, the shell reports the name of the command that
cannot be found.  This is often \texttt{java}, or
\texttt{mskcallcentersim}.
When a class cannot be found, the Java Virtual Machine throws a
\texttt{No\-Class\-Def\-Found\-Error} with the name of the class.
These errors are covered in the HTML pages giving installation
instructions for ContactCenters, on
\ccurl.

\subsection{Unmarshalling errors}

If a parameter file given by the user is not a well-formed or valid
XML document, error messages are displayed, and usually the simulator
aborts.
One then needs to edit the parameter file in order to correct the
errors before the simulator can be run again.
In this subsection, we present some examples of such unmarshalling
errors.

Note that using an XML editor such as
SyncRO Soft Ltd.'s \mbox{$<$oXygen/$>$} or
Altova's XMLSpy may prevent most errors of this type by
verifying the syntax, and validating the parameter files against the
corresponding Schemas.  Opening a parameter file producing an error
with the simulator in such an XML editor may also provide additional
guidance to correct the error.

\subsubsection{Missing ending tag}

Suppose that the simulator is given the following XML file for
experiment parameters.
\begin{lstlisting}[language=XML]
<ccapp:repSimParams minReplications="300"
    xmlns:ccapp="http://www.iro.umontreal.ca/lecuyer/contactcenters/app">
   <report confidenceLevel="0.95">
</ccapp:repSimParams>
\end{lstlisting}
Running the program with this gives the following error message.
\begin{lstlisting}[language={},breaklines,prebreak={\char92}]
The following problem occurred during unmarshalling.
[FATAL ERROR] The element type "report" must be terminated by the matching end-tag "</report>". at file:contactcenters/doc/msk/repSimParams.xml, line 4, column 3
\end{lstlisting}
In the above and all
following error messages, the $\backslash$ character indicates that a line
is broken here while it is not in the message printed by the program.
The error occurs, because at line 4, the current opened elements are
\texttt{rep\-Sim\-Params}, and \texttt{report}.
Trying to close \texttt{rep\-Sim\-Params} before closing
\texttt{report} is of course an error.
The error can be corrected by adding
\lstinline[language=XML]{</report>}
at the beginning of line 4 of the file, just before
\lstinline[language=XML]{</ccapp:repSimParams>}.
Another equivalent correction is to turn \texttt{report} into a
self-closing element, by adding a \texttt{/} symbol before the \texttt{>}.
This reverts back to the parameter file in
Listing~\ref{par:repSimParams}.

\subsubsection{Forgotten closing bracket}

Suppose that we use the following as experiment parameters.
\begin{lstlisting}[language=XML]
<ccapp:repSimParams minReplications="300"
    xmlns:ccapp="http://www.iro.umontreal.ca/lecuyer/contactcenters/app"
   <report confidenceLevel="0.95"/>
</ccapp:repSimParams>
\end{lstlisting}
This leads to the following error message.
\begin{lstlisting}[language={},breaklines,prebreak={\char92}]
The following problem occurred during unmarshalling.
[FATAL ERROR] Element type "ccapp:repSimParams" must be followed by either attribute specifications, ">" or "/>". at file:contactcenters/doc/msk/repSimParams.xml, line 3, column 4
\end{lstlisting}
This indicates that the parser is waiting for a closing bracket while
the user gives a new XML element.
Adding \texttt{>} at the beginning of line 3 solves the problem.
Note that adding \texttt{/>} instead would turn
\texttt{ccapp:repSimParams} into a self-closing element, and trigger
the following error message while parsing the rest of the file.
\begin{lstlisting}[language={},breaklines,prebreak={\char92}]
The following problem occurred during unmarshalling.
[FATAL ERROR] The markup in the document following the root element must be well-formed. at file:contactcenters/doc/msk/repSimParams.xml, line 3, column 5
\end{lstlisting}
The document is not well-formed anymore, because it contains two root
elements: \texttt{ccapp:rep\-Sim\-Params}, and \texttt{report}.

\subsubsection{Missing namespace URI}

Now suppose we give the following parameter file to the simulator.
\begin{lstlisting}[language=XML]
<ccapp:repSimParams minReplications="300">
   <report confidenceLevel="0.95"/>
</ccapp:repSimParams>
\end{lstlisting}
This results in the following error message
\begin{lstlisting}[language={},breaklines,prebreak={\char92}]
The following problem occurred during unmarshalling.
[FATAL ERROR] The prefix "ccapp" for element "ccapp:repSimParams" is not bound. at file:contactcenters/doc/msk/repSimParams.xml, line 1, column 43
\end{lstlisting}
Here, we used a prefix, \texttt{ccapp}, with no associated namespace
URI.
If we remove \texttt{ccapp:} altogether in the opening and closing
element of the parameter file, we obtain another error message:
\begin{lstlisting}[language={},breaklines,prebreak={\char92}]
The following problem occurred during unmarshalling.
[FATAL ERROR] cvc-elt.1: Cannot find the declaration of element repSimParams'. at file:contactcenters/doc/msk/repSimParams.xml, line 1, column 37
\end{lstlisting}
This occurs, because the XML Schema is expecting an element in a
predefined
namespace URI.
Therefore, one must keep the \texttt{ccapp} prefix, and
add the attribute
\begin{lstlisting}[language=XML]
xmlns:ccapp="http://www.iro.umontreal.ca/lecuyer/contactcenters/app"
\end{lstlisting}
in order to bind the prefix to the correct namespace.

\subsubsection{Invalid name of attribute}

Let's take the example in Listing~\ref{par:singleQueue}, and
suppose we change the \texttt{name} attribute of the first
\texttt{service\-Level} element to \texttt{<20s} rather than the
original \texttt{20s}. The service level element would look like
\begin{lstlisting}[language=XML]
   <serviceLevel name="<20s">
      <awt>
         <row>PT20S</row>
      </awt>
      <target>
         <row>0.8</row>
      </target>
   </serviceLevel>
\end{lstlisting}
This gives the following error message.
\begin{lstlisting}[language={},breaklines,prebreak={\char92}]
The following problem occurred during unmarshalling.
[FATAL ERROR] The value of attribute "name" associated with an element type "null" must not contain the '<' character. at file:contactcenters/doc/msk/singleQueue.xml, line 28, column 24
\end{lstlisting}
This error occurs, and confuses the XML parser as well, because the
\texttt{<} character is forbidden in attribute names.
Removing the offending character, or escaping it with
\texttt{\&lt;}, will solve the problem.

\subsubsection{Invalid format for a numeric parameter}

Suppose that we need 95\% confidence intervals in the statistical
report produced by the simulator.
Intuitively, an XML file with the following contents will do the job:
\begin{lstlisting}[language=XML]
<ccapp:repSimParams minReplications="300"
    xmlns:ccapp="http://www.iro.umontreal.ca/lecuyer/contactcenters/app">
   <report confidenceLevel="95%"/>
</ccapp:repSimParams>
\end{lstlisting}
However, this parameter file gives the following error.
\begin{lstlisting}[language={},breaklines,prebreak={\char92}]
The following problem occurred during unmarshalling.
[FATAL ERROR] cvc-datatype-valid.1.2.1: '95%' is not a valid value for 'double'. at file:contactcenters/doc/msk/repSimParams.xml, line 3, column 35
\end{lstlisting}
The file is invalid, because 95\% is not a valid representation for
the number 0.95; one must encode \texttt{0.95} directly in the
parameter file.
Note that other similar error messages will show up if a confidence
level is outside $]0,1[$, if a negative number of replications is
given using the \texttt{min\-Replications} attribute, etc.

\subsubsection{Invalid name of element}

We return to Listing~\ref{par:singleQueue}, and replace
\texttt{inbound\-Type} with \texttt{call\-Type}.
\begin{lstlisting}[language={},breaklines,prebreak={\char92}]
The following problem occurred during unmarshalling.
[FATAL ERROR] cvc-complex-type.2.4.a: Invalid content was found starting with element 'callType'. One of '{properties, busynessGen, inboundType, arrivalProcess, outboundType, dialer, agentGroup}' is expected. at file:contactcenters/doc/msk/singleQueue.xml, line 4, column 34
\end{lstlisting}
This error happens, because \texttt{call\-Type} is not a valid child
for \texttt{MSKCCParams}. Using one of the proposed element names can
solve the problem.
Here, we need to use \texttt{inbound\-Type} to represent our inbound
call type.

\subsection{\texttt{CallCenterCreationException}}

Errors can still occur with well-formed and valid parameter files,
because the XML Schema language does not cover all the constraints
that are imposed on a XML document.
When such an error occurs, several lines of text are displayed, giving
more and more precision about the nature of the problem.
Traversing this chain of error is the best way to diagnose the
problem, and fix the parameter file.

More specifically,
when the model of the call center is constructed, a Java exception
is thrown if some invalid parameter is found.
The exception is caught by higher-level components of the model which
wrap it up in order to add details which are necessary to circumvent
the problem.
The top-level exception for model construction problems is the
\texttt{Call\-Center\-Creation\-Exception}, which is caught by the main
program which displays the error message.
Here, we give some examples of such error messages, with the fix to
the parameter file for each problem.
Note, however, that this is not an exhaustive list of problems.

\subsubsection{Invalid name of probability distribution}

Suppose that, in Listing~\ref{par:singleQueue}, we set the
\texttt{distribution\-Class} attribute of element
\texttt{patience\-Time} to \texttt{Exponential\-From\-Mean} rather
than \texttt{Exponential\-Dist\-From\-Mean}.
Running the simulator with the modified parameter file gives the
following error message.
%
\begin{lstlisting}[language={},breaklines,prebreak={\char92}]
umontreal.iro.lecuyer.contactcenters.msk.model.CallCenterCreationException: Cannot create call factory for call type 0 (Inbound Type)
Caused by umontreal.iro.lecuyer.contactcenters.msk.model.CallFactoryCreationException: Cannot create patience time distribution
Caused by umontreal.iro.lecuyer.xmlbind.DistributionCreationException: The string ExponentialFromMean does not correspond to a fully-qualified class name, or to a class in package umontral.iro.lecuyer.probdist, or it maps to a class not implementing umontreal.iro.lecuyer.probdist.Distribution
Caused by java.lang.ClassNotFoundException: Cannot find the class with name ExponentialFromMean
\end{lstlisting}
%
The first line indicates that the call factory generating calls of
type~0, which correspond to the only call type in the parameter file,
could not be created by the program.
The second line gives clues on the cause of this error:
the distribution of the patience time could not be created.
The third line then gives a clue on the reason why the distribution could
not be created: the distribution class
\texttt{Exponential\-From\-Mean} could not be found
in the \texttt{Probdist} package.
The last line confirms that no class were found, the other possibility
being that a class not implementing the \texttt{Distribution}
interface were referred to by the user.
Looking at the documentation of SSJ, one can see that no class with
that name exists. However, a class with similar name
\texttt{Exponential\-Dist\-From\-Mean} exists.  Using that class name
fixes the parameter file.

\subsubsection{Incorrect number of parameters for a probability distribution}

Suppose now that in the example of Listing~\ref{par:singleQueue},
we put \texttt{100 2} in the \texttt{default\-Gen} child of
\texttt{service\-Time}.
This leads to the following error message.
\begin{lstlisting}[language={},breaklines,prebreak={\char92}]
umontreal.iro.lecuyer.contactcenters.msk.model.CallCenterCreationException: Cannot create call factory for call type 0 (Inbound Type)
Caused by umontreal.iro.lecuyer.contactcenters.msk.model.CallFactoryCreationException: Cannot create service time distribution
Caused by umontreal.iro.lecuyer.xmlbind.DistributionCreationException: Cannot find a suitable constructor; check the number of specified parameters, for distribution class umontreal.iro.lecuyer.probdist.ExponentialDistFromMean with parameters (100.0, 2.0)
\end{lstlisting}
The first two lines of the error message are very similar to the
beginning of the error message in the preceding paragraph.
Here, the error indicates that the service time distribution for the
first (and sole) call type of the model could not be created.
The last line indicates, as an explanation of the error, that no
suitable constructor could be found.
Looking at the documentation of the
\texttt{Exponential\-Dist\-From\-Mean}, we notice that no constructor
taking two arguments are defined.
The problem is thus related to the number of arguments given in
\texttt{default\-Dist}.  Giving two arguments here is of course
incorrect, because the constructor only accepts one, the mean of the
distribution.

\subsubsection{Invalid parameters for a probability distribution}

If the \texttt{default\-Gen} child element contains a negative value
for an exponential distribution, the following error message shows up.
\begin{lstlisting}[language={},breaklines,prebreak={\char92}]
umontreal.iro.lecuyer.contactcenters.msk.model.CallCenterCreationException: Cannot create call factory for call type 0 (Inbound Type)
Caused by umontreal.iro.lecuyer.contactcenters.msk.model.CallFactoryCreationException: Cannot create service time distribution
Caused by umontreal.iro.lecuyer.xmlbind.DistributionCreationException: An error occurred during call to constructor, for distribution class umontreal.iro.lecuyer.probdist.ExponentialDistFromMean with parameters (-100.0)
Caused by java.lang.IllegalArgumentException: lambda <= 0
\end{lstlisting}
The first two lines are the same as the previous error message, but
now, a constructor could be found.
However, according to the third line, an error occurred while it was
called.
The nature of the error is given by the last line: an illegal-argument
exception caused by a negative value.

This kind of error is not caught up at the time of the validation,
because the range of the parameters depends on the specific class of
distribution being used.

\subsubsection{Not enough arrival rates}

Suppose now that the description of the arrival process in
Listing~\ref{par:singleQueue} becomes
\begin{lstlisting}[language=XML]
      <arrivalProcess type="PIECEWISECONSTANTPOISSON" normalize="true">
         <arrivals>100 150 150 180 200 150 150 150 120 100 80 70</arrivals>
      </arrivalProcess>
\end{lstlisting}
The simulator then displays the following error message.
\begin{lstlisting}[language={},breaklines,prebreak={\char92}]
umontreal.iro.lecuyer.contactcenters.msk.model.CallCenterCreationException: Cannot create arrival process for inbound call type 0 (Inbound Type)
Caused by umontreal.iro.lecuyer.contactcenters.msk.model.ArrivalProcessCreationException: An arrival rate is required for each main period
\end{lstlisting}
The first line indicates that the arrival process for the first (and
sole) inbound type of the model could not be created.
This circumvents the problem to the \texttt{arrival\-Process}
element.
The second line specifies that there must be one arrival rate for each
main period in the model.
The attribute \texttt{period\-Duration} indicates 13, while the
\texttt{arrivals} element contains only 12 values, so we need to add
the missing arrival rate in the last main period to fix the parameter file.

\subsubsection{Invalid dimensions of matrix of ranks}

Suppose that we change the \texttt{router} element in
Listing~\ref{par:singleQueue}
to
\begin{lstlisting}[language=XML]
   <router routerPolicy="AGENTSPREF">
     <ranksTG>
       <row>1  1</row>
     </ranksTG>
     <routingTableSources ranksGT="ranksTG"/>
   </router>
\end{lstlisting}
This gives the following error message:
\begin{lstlisting}[language={},breaklines,prebreak={\char92}]
umontreal.iro.lecuyer.contactcenters.msk.model.CallCenterCreationException: Cannot create router
Caused by umontreal.iro.lecuyer.contactcenters.msk.model.RouterCreationException: Error initializing data structures
Caused by java.lang.IllegalArgumentException: Invalid type-to-group matrix of ranks
Caused by java.lang.IllegalArgumentException: The given matrix has 2 columns but it needs 1 columns
\end{lstlisting}
The first line indicates that the router could not be created.
Then, a second line tells that some data structures could not be
initialized properly.
The third line indicates which data structure is invalid, namely the
type-to-group matrix of ranks.
Note that this matrix of priorities is different from the
type-to-group map, which is an array of overflow lists.
The last line of the error message indicates the exact nature of the
problem: wrong number of columns in the given matrix.

\subsection{Execution errors}
\label{sec:exerr}

The Java Virtual Machine may also exit with an error message if
something wrong happens during the simulation.
This can be caused by many factors, ranging from insufficient memory
to bugs in the simulator.
Here, we list the most common execution errors.

\subsubsection{\texttt{OutOfMemoryError}}

This happens if the simulator runs out of memory because of a large
simulated model.
If the simulator or any Java program using the simulator
exit with an \texttt{Out\-Of\-Memory\-Error}, the
maximal size of the Java heap can often be increased by using a JVM
option on the command-line used to start the program.  For Sun's JRE,
the option is \texttt{-Xmx}.  One must use the \texttt{CCJVMOPT}
environment variable to increase the heap size for the ContactCenters'
scripts.
For example, the following line sets the heap size to 800 megabytes
for any subsequent call to \texttt{mskcallcentersim}, when
using Bourne shell under UNIX/Linux:

\noindent\texttt{export CCJVMOPT=-Xmx800m}

This can also be tried if the simulator
seems to be excessively slow, because increasing the maximal heap size
will reduce the use of the garbage collector.
However, the maximal heap size should not be near or larger than
the total amount
of system memory.  Otherwise, the operating system would use hard disk
as virtual memory, which would slow down the simulator as well as any
other running applications significantly.

It often happens that increasing the heap size does not remove
the error, because
memory problems can also arise if the model suffers from some defects
such as no agent in any groups, too long patience times,
or an ill-designed routing policy.

A bad interpretation of the arrival rates may cause the program to run
out of memory.
By default, the arrival rates are interpreted relative to one
simulation time unit.
For example, in Listing~\ref{par:singleQueue},
if we omit the \texttt{normalize} attribute of the
\texttt{arrival\-Process} element,
the arrival rates will give the expected number of arrivals during one
second.
An average of 100 arrivals per second result in 360,000 arrivals per
time period, which is excessively high.
Setting \texttt{normalize} to \texttt{true} instructs the simulator to
interpret the arrival rates relative to one hour, so 100 now means
average 100
arrivals per hour.

\subsubsection{\texttt{IOException}}

This error may occur at the end of the simulation, when results are
written to disk.
The most common causes of this error are lack of disk space or no
permission on the file system to write the results.
Trying to save results at a different place on the file system, or on
another drive, may solve the problem.

\subsubsection{Warnings about detailed agent groups followed by an
  \texttt{Illegal\-State\-Exception}}

Some routing policies, e.g., \texttt{AGENTSPREF}
(see Section
\ref{javadoc:umontreal.iro.lecuyer.contactcenters.app.RouterPolicyType}),
require that the
agent groups be in detailed mode, i.e., that they consider each agent
as a separate object rather than simply updating counters for the
number of busy and idle agents in the group.
If the agent groups are not in detailed mode, a warning is printed
before the simulation starts.
Sometimes, the router does not need the information about detailed
agent groups, so the simulation succeeds even if the user gets a
warning.
However, if a routing decision requires information about the agents
in a group, e.g., for computing the longest idle time, an
\texttt{Illegal\-State\-Exception} is thrown and the simulation
aborts.

The simplest solution to this problem is to switch agent groups to
detailed mode, by setting the \texttt{detailed} attribute to
\texttt{true} in all \texttt{agent\-Group} elements of the
XML parameter file for the model.
Alternatively, one may use a different routing policy not needing
detailed agent groups.
For some routing policies, it may also be possible to change
parameters in such a way that longest idle times are not needed.
See the end of Section~\ref{sec:addAgentGroup} for an example of this.

\subsubsection{Infinite loops}

Some model parameters have caused infinite loops
during simulation.
Although these problems have been fixed, this could occur again with
other model parameters.
If  such a bug occurs, one should send the parameter file causing the
error to the author of ContactCenters.

When simulation is done with batch means, infinite loops are often
caused by the heuristic which initializes the system to a non-empty
state.
Turning the \texttt{init\-Non\-Empty} attribute to \texttt{false} in
experiment parameters might work around the problem.

\subsubsection{\texttt{Null\-Pointer\-Exception} and other exceptions}

This error is often caused by a bug in the simulator.
If this happens, contact the author with the complete stack trace
which is displayed, and the circumstances of the error.
A sample parameter file as small as possible
but causing the error
on a regular basis might be very helpful for
the author to diagnose such problems.

\subsubsection{Slow simulation}

The main factors affecting the performance of the simulator are
the size of the simulated model, and the computed statistics.
Moreover, if the simulator uses too much memory, the garbage collector
of Java is triggered more often, which slows down the simulation.
Increasing the maximal heap size using the \texttt{-Xmx} JVM option
(see preceding section) can sometimes help.

% A common cause of slow simulation is an excessive arrival rate for one
% or more call types.
% For example, by default, the simulator interprets arrival rates given
% in the
% \texttt{arrivals} element of \texttt{arrival\-Process} as
% the expected number of arrivals per time unit.
% Giving an expected number of arrivals during a full time period of
% course results in very high arrival rates.
% In that common case, one should normalize the entered arrival rates or
% use the \texttt{normalize} attribute of \texttt{arrival\-Process}.

The routing policy can also affect performance.
Using complex policies taking priorities and delays into account
requires more computation than a simple policy based on a fixed list
of call types and agent groups.
Enabling advanced model aspects such as call transfers and virtual
queueing can also decrease performance, since these aspects require
calls to be processed more than once by the system.

Performance can often be increased by restricting the set of computed
statistics.
By default, the simulator estimates a large quantity of performance
measures, but one often does not need all these estimates.
One can restrict to a fixed set of performance measures by setting
the \texttt{restrict\-To\-Printed\-Stat} attribute
to \texttt{true}
in experiment parameters.
For example, this attribute may be used in a parameter file such as
the one shown on Listing~\ref{par:repSimParamsStat}.
