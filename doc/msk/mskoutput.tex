\section{The output of the simulator}
\label{sec:mskoutput}

Any report produced by the simulator contains the following elements:
\begin{itemize}
\item General information about the experiment
\item Statistics for each selected performance measure
\end{itemize}

\subsection{The contents of a report}

General information includes the names of the parameter files, the
date at which the experiment started, the CPU time required to carry
out the complete simulation, the sample size, etc.  This information
is presented in the form of one (key: value) pair per line.
The main part of the report contains statistics for performance
measures.

Several quantities are computed during the simulation: event
counts, average times, and integrals.  Each time a call exits
the system, counters are updated to keep track of various quantities.
All these random variables can be regrouped into a random vector
$\boldX=(X_0,\ldots,X_{d-1})$ we will call
an \emph{observation}.  Using some experimental techniques
presented in section~\ref{sec:mskexp}, the simulator can obtain $n$
copies of $\boldX$ which is called a \emph{sample}.  Statistics are
computed by applying some functions on this sample
$\boldX_0,\ldots,\boldX_{n-1}$, where $\boldX_r=(X_{0, r}, \ldots,
X_{d-1, r})$ is the $r$th observation.
Let
\[\bar{\boldX}_n=(\bar X_{0, n}, \ldots, \bar X_{d-1, n})=\frac1n\sum_{r=0}^{n-1}\boldX_r\]
the \emph{average} of $\boldX_r$,
which is used to estimate the expected value of $\boldX$, denoted as
\[\E[\boldX]=(\E[X_0], \ldots, \E[X_{d-1}]).\]
The vector
$\bar{\boldX}_n$ is an unbiased estimator of $\E[\boldX]$ if the
observations are independent and identically distributed (i.i.d.).
We are also interested in functions $g(\boldX)$ such as ratios.
The function of averages $g(\bar{\boldX}_n)$ is used to estimate the
function
of expectations $g(\E[\boldX])$.  This estimator is biased, unless
$g(\boldX)$ is a linear function of $\boldX$.
Subsection~\ref{javadoc:umontreal.iro.lecuyer.contactcenters.app.PerformanceMeasureType}
explains how these quantities are regrouped in matrices for easier reporting.

For each performance measure, i.e., element $\E[X_j]$ or function
$g(\E[\boldX])$, the simulator outputs the following
statistics (all statistics are of course undefined when $n=0$):
\begin{description}
\item[Minimum] The minimal value of $X_j$ among all observations.  No
  minimum is available for functions of multiple averages.
\item[Maximum] The maximal value of $X_j$ among all observations.  No
  maximum is available for functions of multiple averages.
\item[Average] The average $\bar X_{j, n}$, or function of averages
  $g(\bar{\boldX}_n)$, of the
  observations.
\item[Standard deviation] The sample standard deviation of the observations,
  i.e., $\sqrt{\Var[X_j]}$ or $\sqrt{n\Var[g(\bar{\boldX}_n)]}$.  This
  corresponds to the asymptotic standard deviation in the case of a
  function of several averages.
  The value is undefined if $n<2$.
\item[Confidence interval] An interval $[a, b]$ containing $\E[X_j]$ (or
  $g(\E[\boldX])$) with probability $1-\alpha$, where $1-\alpha$
  is a \emph{confidence level} that can be adjusted via a simulation
  parameter.
  The interval is computed using the normality assumption.
  The value is undefined if $n<2$.
\end{description}
Using the element \texttt{printed\-Stat\-Params} in the parameters of
the report, %(see
%section~\ref{javadoc:umontreal.iro.lecuyer.contactcenters.app.PrintedStatParams})
one can decide which performance measures appear in the report, and
determine if all the statistics or only the averages are needed.

We now explain the last two statistics in more details.
Let
\[
\boldS_n = \frac1{n-1}\sum_{r=0}^{n-1} (\boldX_r -
\bar{\boldX}_n)^\tr(\boldX_r - \bar{\boldX}_n),
\]
be the \emph{sample covariance} of the $\boldX_r$'s,
which is used to estimate the covariance matrix
\[\boldSigma = \E[(\boldX - \E[\boldX])^\tr(\boldX - \E[\boldX])].\]
The simulator computes only parts of this sample covariance matrix, in
particular
elements $(j, j)$ estimating $\Var[X_j]=n\Var[\bar X_{j, n}]$,
which can be used to estimate the error on $\bar X_{j,
  n}$.
% To estimate the error on $\bar X_{j, n}$ for $j=0,\ldots,d-1$, the
% simulator computes the \emph{sample variance}
% \[S_{X_j, n}^2=\frac1{n-1}\sum_{r=0}^{n-1} (X_{j, r} - \bar X_{j,
%   n})^2\]
% which estimates the variance of the random variate $X_j$, defined as
% \[\Var(X_j)=\E[X_j^2] - \E[X_j]^2.\]

The sample variance is used to compute a confidence interval on the
true mean $\E[X_j]$ for any $j\in\{0, \ldots, d-1\}$.  Assuming that
$X_{j, r}$ follows the normal distribution,
\[\sqrt{n}(\bar X_{j, n} - \E[X_j])/S_{X_j, n}\]
follows the Student-$t$ distribution with $n-1$ degrees of freedom.
Here, $S_{X_j, n}$ is the sample standard deviation of $X_j$.
If the desired probability that this (random)
interval covers the true mean $\E[X_j]$ (a constant) is $1-\alpha$,
the interval is given by $\bar X_{j, n}\pm t_{n-1, 1-\alpha/2}S_{X_j,
  n}/\sqrt{n}$, where $t_{n-1,1-\alpha/2}$ is the inverse of the
Student-$t$ distribution function with $n-1$ degrees of freedom,
evaluated at $1-\alpha/2$.

Confidence intervals on functions of means are computed
using the delta theorem \cite{tSER80a}.
Here, we explain the special case of ratios used by the simulator.
One can refer to \cite{tSER80a,sLEC06a,iBUI05b} for the general case.
Let $(X,Y)$ be a random vector for which the simulator can
generate a sample $((X_0, Y_0), \ldots, (X_{n-1}, Y_{n-1}))$.  Let
$\bar X_n$ be the average for $X$ and $\bar Y_n$ the average for $Y$;
these quantities estimate $\mu_1=\E[X]$, and $\mu_2=\E[Y]$,
respectively.
Then, the function $\bar\nu_n=\bar X_n/\bar Y_n$
estimates the ratio of means $\nu=\mu_1/\mu_2$.  By a Taylor expansion
of the ratio of averages,
%if $(\bar X_n, \bar Y_n)$ is a multinormal vector,
the asymptotic variance of $\bar X_n/\bar Y_n$, i.e., the variance
when $n$ is large, is given by $\sigma^2/n$, where
\[\sigma^2=(\Var[X] + \nu^2\Var[Y] - 2\nu\Cov[X, Y])/\mu_2^2.\]
The variance
$\sigma^2$ can be estimated by using sample means, variances and covariance.
%as follows:
%\[\hat\sigma_n^2=(S_{X,n}^2+\bar\nu_n^2S_{Y,n}^2
%-2\bar\nu_nS_{XY, n})/\bar Y_n^2\]
%where $S_{X, n}$ estimates the variance of $X$, $S_{Y, n}$ estimates
%the variance of $Y$, and $S_{XY, n}$ estimates the covariance of $X$
%and $Y$.
%\[S_{XY, n}=\frac1{n-1}\sum_{r=0}^{n-1} (X_r - \bar X_n)(Y_r - \bar
%Y_n).\]

Assuming that $(\bar X_n, \bar Y_n)$ follows the multinormal
distribution,
the confidence interval on the ratio of expectations with confidence
level $1-\alpha$ is given by $\bar\nu_n\pm
z_{1-\alpha/2}\sigma_n/\sqrt{n}$, where
$\Phi(z_{1-\alpha/2})=1-\alpha/2$, $\Phi(x)$ being the distribution
function of a standard normal variable.

Note that each confidence interval is computed for a single mean or
ratio of means, independently of other performance measures of the
system.  As a result, if $d>1$ output values are analyzed
simultaneously, the confidence level of the $d$ intervals
is $1-d\alpha<1-\alpha$.
The confidence level for individual performance measures must then be
higher to get the same overall confidence level.

The value of $0/0$ is usually undefined and assigned the NaN (Not a Number) flag.
However, in some ratios, $0/0$ can have some meaning.
In our simulator, $0/0$ is defined as 0 for most performance measures except service level.
For example, if there is no arrival, it is sensible to set the
abandonment ratio and the average waiting time to 0, and
the service level to 1.

For expectations of ratios, $0/0$ observations are not collected,
because fixing an arbitrary value would result in biased estimators.  As a
result, the average is made on less observations, and the average is
NaN if all observations are rejected.

\subsection{The format of the report}
\label{sec:reportformat}

At this moment, four file formats are supported for reports:
XML, plain text, \LaTeX, and
Microsoft Excel.
The first format is intended to be readable by programs while
the last three formats are human-readable.

\subsubsection{Program-readable format}

The XML format is intended to be parsed by Java programs
using the ContactCenters library.
It could also be parsed and processed by any other program
compatible with XML.
Produced XML output files have root element
\texttt{Contact\-Center\-Sim\-Results}, in
namespace URI
\path{http://www.iro.umontreal.ca/lecuyer/contactcenters/app}.
The XML schema for output files can be found in the \texttt{schemas}
subdirectory of ContactCenters, and HTML documentation is available
in \path{doc/schemas}.
%The binary format, using Java
%serialization, may be faster to read, but the XML format may be
%more easily
%reusable by other programs not using ContactCenters, even not using
%Java.

Alternatively, the program can export to a XML file
and compress the file using GZip to save
disk space.
This can be done by giving a file name with the \texttt{.xml.gz}
extension rather than \texttt{.xml}.

\subsubsection{Plain text}

When exporting to plain text, the simulator uses the
platform-default character encoding and line separator.
As a result, the created text file can be opened in any text editor
such as Notepad, GNU Emacs, etc.
After the general information, the report contains a table of summary
results, e.g., the performance measures concerning every call type,
agent group, and period.  Then, for each group of performance
measures, a table of detailed results appear in the report.
Note that the formatting of numbers is locale-specific.  For example,
if the current locale is set to French, the decimal separator
is the comma while the separator is a period for the US locale.

The \LaTeX\ output, which is also plain text but with formatting
instructions,
is intended to be processed by \LaTeX\ to generate printable tables of
results.

\subsubsection{Microsoft Excel}

The Microsoft Excel format is used to transfer results to spreadsheets
for further analysis and reporting.
One is not restricted to Microsoft Excel since
many other spreadsheets, e.g., OpenOffice.org and KOffice,
can read and write Excel files.

The Excel report is divided in at most three sheets.
The first sheet provides summary information only: the general
information, and summary statistics, i.e., statistics for aggregate
performance measures.
The latter are split in two groups: source-related (or call-related)
statistics and destination-related (agent-related and waiting
queue-related) statistics.
The second sheet provides a detailed report for all time-aggregate
performance measures.  This includes, e.g., the service level for
each individual call type but not for each period.
The last sheet contains a detailed report for all performance
measures, including statistics for individual periods.

\subsubsection{Localized format for reports}

Some aspects of the reports produced by \texttt{mskcallcentersim}
depend on the host environment of the JVM.
These aspects include the character encoding of the report,
the line delimiters, strings describing types of performance measures,
and the format of the numbers.
The last two elements are influenced by the locale of the virtual
machine which executes the simulator, which corresponds to the default
locale of the host environment.
At this moment, only English and French are supported as languages for
reports. If the current locale corresponds to another language,
all text in the report will be in English, with locale-specific
formatting for numbers.

Some of these aspects can be customized using OS-specific options.
For example, calling \texttt{LC\_ALL=en\_US.UTF-8 mskcallcentersim}
launches the simulator with the US English locale, and UTF-8 encoding;
this is the default for most UNIX/Linux distributions.
However, on other operating systems, such as Microsoft's Windows,
there is no built-in way to alter the default locale for a given
program without changing the system-wide regional settings.
However, Java properties can be changed to alter the default locale,
and other parameters, in a platform-independent way.
Table~\ref{tab:repjavaprop} lists such properties.
The properties can be modified through the \texttt{-D} JVM option.
For example, setting the \texttt{CCJVMOPT} environment variable
to \texttt{-Duser.language=en} sets the language of reports to
English.

\begin{table}
\caption{Most common Java properties affecting reporting}
\label{tab:repjavaprop}

\centering\begin{tabular}{|l|ll|}\hline
Property & Action & Sample value \\ \hline
\texttt{user.language} & Language of strings in reports & \texttt{en}
\\
\texttt{user.region} & Region affecting number and date format &
\texttt{US} \\
\texttt{file.encoding}  & Character encoding for reports &
\texttt{UTF-8} \\
\hline
\end{tabular}
\end{table}
