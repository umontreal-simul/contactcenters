\part{Tutorial}
\section{Overview}

A \emph{contact center} is a set of resources (communication
equipment, employees, computers, etc.) providing an interface between
customers and a business \cite{ccMEH03a,ccGAN02a,ccAVR05b,ccAKS07a}.
Each \emph{contact} represents a customer reaching the contact center
to obtain some form of service.  The service is made by employees in
the contact centers called \emph{agents}.  Each agent is a member of
an \emph{agent group} which determines its characteristics
(skills, speed of service, etc.).
When a contact cannot be served immediately, it is put in a
\emph{waiting queue} to be served later.  The contact center
components are linked together by a \emph{router} which decides on how
to assign calls to agents.  A \emph{call center} is a
special form of contact center where each contact corresponds to a
telephone call.

The ContactCenters library is built using the Java
programming language \cite{iGOS00a}
and the Stochastic Simulation in Java (SSJ) library \cite{iLEC04j},
and permits one to implement
simulators for
contact centers.  The library provides building blocks such
as classes representing the contacts in the center, the agent groups,
the waiting queues, and the router.  The programmer combines these
blocks to make a simulator.
However, creating a simulator
directly using
this library involves Java programming.

This document presents a
ready-to-use generic simulator for the particular case of a blend and
multi-skill call center with multiple call types, agent groups and
simulation periods.  It can simulate \emph{inbound} calls arriving in
the system following a stochastic arrival process as well
as \emph{outbound} calls made by predictive dialers.  Service and
patience times are also random, and come from
any
probability distribution supported by SSJ, and parameters can change
from time periods to periods.

This simulator is configured
through XML files. Compilation of Java
code is not required, except if the simulator has to be extended,
or used internally by another program.
%  A XML file represents an hierarchical document composed of
%a root element which can have attributes as well as nested text and
%children elements.  Such a file can be
%created and modified with any text editor, but specialized editors
%dedicated to XML, e.g., XML Spy, are also available.
%  While XML specifies how to
%format elements and attributes in such a file,
%schemas indicate
%which structures are allowed for describing a specific concept.
Any XML document intended to be processed by a program conforms to a
schema.
The simulator uses one such schema for the
parameters of the simulated model, and a
second schema for the parameters of the experiment method.

% However, this simulator prevents the user from accessing all the
% possibilities of ContactCenters. If custom arrival processes, routing
% or dialing policies are needed, creating a full Java program using
% ContactCenters may be easier than trying to adapt the complex generic
% simulator.

The rest of this document is organized as follows.
In the next section, we present the call center model
implemented by the generic simulator.  We define the structure of
possible call centers as well as the supported types
of experiments.
Section~\ref{sec:mskconfig} introduces the format of the configuration
files for the simulator by some commented examples.  This is
a good way to learn how to make configuration files, not a
reference documentation.
Section~\ref{sec:mskuse} demonstrates how to run the
simulator from the command-line while section~\ref{sec:mskusejava}
shows how to interact with the simulator from a Java program.
Section~\ref{sec:msktrouble} discusses most common problems encountered
when using the simulator.
The last sections contain a reference manual
providing detailed documentation for each supported performance measure,
routing policies, dialing policies, arrival processes, the
supported types of experiments, and the format of generated reports.

Section~\ref{sec:mskxml}
gives a primer on XML, and the data types used in the
parameter files.
It also gives some examples on how parameter-specific documentation,
which is available in HTML only, can be retrieved.
The documentation for
each parameter was generated from the annotations in the
corresponding XML schemas, and can be located
in the \path{doc/schemas} subdirectory of ContactCenters.
