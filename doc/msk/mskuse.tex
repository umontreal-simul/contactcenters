\section{Running simulations from the command line}
\label{sec:mskuse}

This section gives some examples on how to use the blend/multi-skill
call center simulator from the command-line.
This requires a Java Virtual Machine to be installed, and the
environment to be set up to access the Java code for the simulator.
See \ccurl{} for more information
about installing ContactCenters.

\subsection{Calling the generic simulator from the command-line}
\label{sec:cmdline}

The simplest way to invoke the simulator is from the command-line,
e.g., the DOS prompt under Microsoft Windows or a shell under
UNIX/Linux.  The
simulator requires two file names: the call
center and simulation parameter files.  It reads these
files, performs the simulation, and prints a full statistical report.
The file for call center parameters must contain a root element named
\texttt{MSKCCParams}.
% (see
%section~\ref{javadoc:umontreal.iro.lecuyer.contactcenters.msk.CallCenterParams}).
The root element for simulation parameters is
\texttt{batch\-Sim\-Params} to use batch means
%(see
%Section~\ref{javadoc:umontreal.iro.lecuyer.contactcenters.app.BatchSimParams}),
or \texttt{rep\-Sim\-Params} to use independent replications.
%(see Section~\ref{javadoc:umontreal.iro.lecuyer.contactcenters.app.RepSimParams}).
For example, the simulator can be launched with the following
command.

%\noindent\texttt{java
%  umontreal.iro.lecuyer.contactcenters.msk.CallCenterSim}\\
%\texttt{mskccParamsThreeTypes.xml batchSimParams.xml}
\noindent\texttt{mskcallcentersim}
 \texttt{ mskccParamsThreeTypes.xml \ batchSimParams.xml}

To run a non-stationary simulation, a file containing parameters for an
experiment using independent replications needs to be given instead of
a file with parameters for batch means.
The (long) statistical report is printed on the standard output and
contains results for every supported performance measure, unless
specific measures are selected via the experiment parameters,
through the
\texttt{printed\-Stat} element.
%(see
%p.~\pageref{javadoc:umontreal.iro.lecuyer.contactcenters.app.ReportParams:getPrintedStatParams()}).
The first part of the report contains a table giving the aggregate
performance measures whereas the second part gives all elements of the
matrices of results.
Listing~\ref{lst:exout} presents an example of such a report.

\begin{lstlisting}[language={}, caption={Sample output of the
    simulator}, label=lst:exout]
Call center parameter file       singleQueue.xml
Simulation parameter file        repSimParamsStat.xml
Experiment started on            May 9, 2008 3:02:30 PM
Number of simulated replications 300
Total CPU time                   0:0:3.88

Aggregate performance measures
                  | Values
                  | Min Max Average Std. Dev. Conf. Int.
------------------------------------------------------------
M Service level   | --- --- 94.4%   1.69%     [94.2%, 94.6%]
e Occupancy ratio | --- --- 55.8%   1.68%     [55.7%, 56%]
a
s
u
r
e
s

Service level
                                    | Values
                                    | Min Max Average Std. Dev. Conf. Int.
------------------------------------------------------------------------------
T Inbound Type, sl 20s, 8:00:00 AM  | --- --- 78.1%   11.2%     [76.8%, 79.4%]
y Inbound Type, sl 20s, 9:00:00 AM  | --- --- 85.9%   9.17%     [84.8%, 86.9%]
p Inbound Type, sl 20s, 10:00:00 AM | --- --- 97.5%   3.05%     [97.1%, 97.8%]
e Inbound Type, sl 20s, 11:00:00 AM | --- --- 93.7%   5.96%     [93.1%, 94.4%]
s Inbound Type, sl 20s, 12:00:00 PM | --- --- 90.1%   7.33%     [89.3%, 91%]
  Inbound Type, sl 20s, 1:00:00 PM  | --- --- 93%     6.41%     [92.3%, 93.8%]
  Inbound Type, sl 20s, 2:00:00 PM  | --- --- 97.4%   3.21%     [97.1%, 97.8%]
  Inbound Type, sl 20s, 3:00:00 PM  | --- --- 97.3%   3.36%     [97%, 97.7%]
  Inbound Type, sl 20s, 4:00:00 PM  | --- --- 93.5%   5.61%     [92.9%, 94.2%]
  Inbound Type, sl 20s, 5:00:00 PM  | --- --- 97%     3.92%     [96.6%, 97.5%]
  Inbound Type, sl 20s, 6:00:00 PM  | --- --- 87.3%   9.99%     [86.2%, 88.4%]
  Inbound Type, sl 20s, 7:00:00 PM  | --- --- 91.8%   7.64%     [91%, 92.7%]
  Inbound Type, sl 20s, 8:00:00 PM  | --- --- 94.7%   5.73%     [94%, 95.3%]
  Inbound Type, sl 20s, all periods | --- --- 92.5%   1.90%     [92.2%, 92.7%]
  Inbound Type, sl 30s, 8:00:00 AM  | --- --- 81.8%   10.7%     [80.6%, 83%]
  Inbound Type, sl 30s, 9:00:00 AM  | --- --- 89.2%   8.08%     [88.3%, 90.1%]
  Inbound Type, sl 30s, 10:00:00 AM | --- --- 98.4%   2.45%     [98.1%, 98.7%]
  Inbound Type, sl 30s, 11:00:00 AM | --- --- 95.7%   5.03%     [95.1%, 96.2%]
  Inbound Type, sl 30s, 12:00:00 PM | --- --- 93%     6.30%     [92.3%, 93.7%]
  Inbound Type, sl 30s, 1:00:00 PM  | --- --- 94.9%   5.59%     [94.3%, 95.6%]
  Inbound Type, sl 30s, 2:00:00 PM  | --- --- 98.4%   2.58%     [98.1%, 98.7%]
  Inbound Type, sl 30s, 3:00:00 PM  | --- --- 98.3%   2.85%     [97.9%, 98.6%]
  Inbound Type, sl 30s, 4:00:00 PM  | --- --- 95.4%   4.75%     [94.8%, 95.9%]
  Inbound Type, sl 30s, 5:00:00 PM  | --- --- 97.9%   3.33%     [97.6%, 98.3%]
  Inbound Type, sl 30s, 6:00:00 PM  | --- --- 89.6%   9.26%     [88.6%, 90.7%]
  Inbound Type, sl 30s, 7:00:00 PM  | --- --- 93.5%   6.80%     [92.7%, 94.3%]
  Inbound Type, sl 30s, 8:00:00 PM  | --- --- 95.8%   5.17%     [95.2%, 96.4%]
  Inbound Type, sl 30s, all periods | --- --- 94.4%   1.69%     [94.2%, 94.6%]
\end{lstlisting}

The stationary simulation is divided in several steps.
The non-stationary simulation uses a similar logic, except that there
is no system initialization and warmup, and independent replications
are simulated instead of batches.

\begin{enumerate}
\item Parameters are read from the given XML files and verified for
  validity using XML schemas.  An error message is printed in case of
  a problem.
\item If the initialization is non-empty, i.e., the
  \texttt{init\-Non\-Empty} attribute is \texttt{true} in parameter
  file, the system starts simulating
  arrivals until the number of busy agents is sufficiently high,
  discarding any calls which cannot be served due to unavailable free
  agents.  No services end during this initialization period.
\item The simulation starts with a warmup period with fixed
  duration.  No statistical observations are collected during this
  period.
\item The target number of batches is initialized to the minimal
  number of batches.
\item Simulate until the target number of batches is available.
% \item The following steps are repeated until the target relative error is
%   reached, \texttt{max\-Batches} real batches are simulated,
%   or only once if sequential sampling is not used.
%   \begin{enumerate}
%   \item Simulate until the target number of batches is available.
%   \item Compute the estimators for checked
%     performance measures as well as confidence
%   intervals to estimate the relative error.
%   \item If no sequential sampling is used, or if the relative error
%     is smaller than the target relative error, stop the simulation.
%   \item Estimate the number of additional batches needed to reach the
%   target error and continue the simulation.
%   \end{enumerate}
\item Print the statistical report.
\end{enumerate}

\subsubsection{Calling the CTMC simulator}

The simplified simulator using the CTMC model can be called from the
command-line by using the \texttt{ctmccallcentersimmp} command.
This command is similar to \texttt{mskcallcentersim}, except that the
parameter file for the experiment must contain a root element with
name \texttt{ctmcrep\-Sim\-Params}.

A simpler simulator is available for a single period simulated for
some fixed time duration.  This simulator can be accessed using the
\texttt{ctmccallcentersim} command.

\subsubsection{Passing options to the JVM}

Any option given after the command \texttt{mskcallcentersim} is passed
to the simulator, not to the JVM. One must call \texttt{java}
directly or  set the \texttt{CCJVMOPT}
environment variable (for \emph{ContactCenters Java Virtual Machines
  Options}) to pass options to the JVM.
For example, the following code can be used to enable assertion
checking on UNIX/Linux, using the Bourne shell.

{\noindent\ttfamily
CCJVMOPT="-ea" mskcallcentersim mskccParamsThreeTypes.xml batchSimParams.xml
}

or

{\noindent\ttfamily
export CCJVMOPT="-ea"\\
mskcallcentersim mskccParamsThreeTypes.xml batchSimParams.xml
}

For Microsoft Windows, one can use

{\noindent\ttfamily
set CCJVMOPT="-ea"\\
mskcallcentersim mskccParamsThreeTypes.xml batchSimParams.xml
}

Another JVM option can be used to increase the maximal heap size,
when the simulator throws  an \texttt{Out\-Of\-Memory\-Error}.
See Section~\ref{sec:exerr} for more information.

JVM options are also used to set properties for the Java program, by
passing the \texttt{-D} argument to the JVM.
In particular, the simulator supports the \texttt{cc.noprogressbar}
property which can be used to prevent the simulator
from displaying the progress bar. On UNIX/Linux, this can be done as follows:

{\noindent\ttfamily
CCJVMOPT="-Dcc.noprogressbar"\\
mskcallcentersim mskccParamsThreeTypes.xml batchSimParams.xml
}

or

{\noindent\ttfamily
export CCJVMOPT="-Dcc.noprogressbar"\\
mskcallcentersim mskccParamsThreeTypes.xml batchSimParams.xml
}

For Microsoft Windows, one can use

{\noindent\ttfamily
set CCJVMOPT="-Dcc.noprogressbar"\\
mskcallcentersim mskccParamsThreeTypes.xml batchSimParams.xml
}

Moreover, as we will see in Section~\ref{sec:reportformat},
properties can be used to set the locale of the program in a
platform-independent way.

\subsection{Exporting the statistical report}

Running the
simulator from the command-line is a good way to test parameter files
and get a first idea on the simulation results.
However, for further analysis and comparison of scenarios, exporting
the obtained results is a necessity.
A first idea for this is to use the
operating system's redirection to export the output of
the simulator into a plain text file.
This can usually be done by appending
\texttt{$>$ output.txt} to the command-line, where \texttt{output.txt}
is the name of the output file.
% This works under Windows and Linux, I do not know for Mac.
However, the resulting text file may be long, finding performance
measures of interest may be hard, and further processing requires
parsing the file, which can be difficult and not recommended
as its
format could change in the future.

The \texttt{mskcallcentersim} program supports an optional argument
allowing the statistical results to be exported into a file rather
than displayed on-screen.
At this moment, four file formats are supported: plain text, \LaTeX,
XML, and
Microsoft Excel.
See Section~\ref{sec:reportformat} for more information on these
reports.

Exporting to plain text is equivalent to using
redirection except it does not rely on an operating system specific
syntax. The enhanced command-line for file exportation is

\noindent\texttt{mskcallcentersim} \texttt{ mskccParamsThreeTypes.xml
 \ batchSimParams.xml \ output.ext}

The file format is selected based on the extension \texttt{ext} of the
output file.  The \texttt{txt} extension instructs the program to
create a text file.
%If \texttt{ext} is set to \texttt{bin}, a binary file is
%produced.
If \texttt{ext} is set to \texttt{xml}, one obtains an XML file.
With \texttt{ext} set to \texttt{xml.gz}, the program produces a XML
file, but compresses it using GZip; this can be used to save space
when simulating a large number of scenarios and storing results in XML
files for future processing.
If \texttt{ext} is set to \texttt{tex}, the output file is in a format
suitable to be included into a \LaTeX\ document.
If \texttt{ext} is \texttt{xls}, the program produces an Excel
document.
For any other extension, a plain text file is created after a warning
is displayed.

\subsubsection{Case sensitivity}

Note that the file extensions are case sensitive.
For example, if the name of the output file is
\texttt{output.XLS}, the program will produce a plain text file, even
though \texttt{XLS} extension seems to refer to Excel.
The correct file name is \texttt{output.xls}.

\subsubsection{Exporting versus redirection}

Note that if the redirection operator \texttt{>} is used before the
file name, the file extension is ignored, and a plain text file is
produced.
For example, \texttt{mskcallcentersim singleQueue.xml repSimParams.xml
> output.xls} will produce a text file named \texttt{output.xls}, not
an Excel document.
Removing the \texttt{>} character results in an Excel document.
This is due to the fact that with redirection, the program does not
know the name of the output file.

\subsubsection{Existing output file}

The behavior of the simulator when the specified output file already
exists depends on the format.
The general rule is to avoid destructive overwriting.
Whenever possible, the new output is appended to the old file;
new text is added to text files, and new sheets to Excel workbooks.
For XML files, a warning message is displayed before a file
with a name of the form \texttt{nameN.ext} is created.
Here, \texttt{name} and \texttt{ext} are the original name and
extension while \texttt{N} is a number.
If redirection is used instead of giving a file name to the simulator,
the behavior depends on the operating system, which usually overwrites
existing files.

One can use the \texttt{loadsimres} program to load back a file
containing results, and save it to another format.
The first argument for the program gives the
name of the binary or XML file to load while the second,
optional argument, provides the name of the output file.
If the latter argument is omitted, the results are printed on-screen.
The loading mechanism cannot import text and Excel files.

\subsection{Getting estimated parameters}
\label{sec:estparprg}

For XML files containing data to estimate probability distributions
from, the \texttt{mskestpar} program can be used to convert the data
into parameters from the command line.  For example, the following
command line converts the XML file presented in
Section~\ref{sec:singleQueueMLE} into a file with estimated parameters.

\noindent\texttt{mskestpar singleQueueMLE.xml singleQueueMLEOut.xml}

\subsection{Converting old parameter files}

The \texttt{oldmskccparamsconverter} and
\texttt{oldsimparamsconverter} programs can be used
to convert model and experiment parameter files from
the old format used by ContactCenters 0.8, to the new format
used by ContactCenters 0.9.
These can be used as follows.

\noindent\texttt{oldmskccparamsconverter singleQueueOld.xml singleQueueNew.xml}

This converts a file named  \texttt{mskCCOld.xml} containing
model parameters into a file with name
\texttt{mskCCNew.xml} usable by ContactCenters 0.9.

\noindent\texttt{oldsimparamsconverter repSimParamsOld.xml repSimParamsNew.xml}

This converts a file named \texttt{repSimParamsOld.xml} containing
experiment parameters into a file with name
\texttt{repSimParamsNew.xml} usable by ContactCenters 0.9.
