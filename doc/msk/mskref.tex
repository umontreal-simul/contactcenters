\part{Reference documentation}
\section{Overview}
\label{sec:mskref}

This part contains the reference documentation for how parameters
need to be formatted as well as the output of the
simulator.  It
does not explain how to access the simulator from Java code.  One
needs to refer to Section~\ref{sec:mskusejava}, and
the ContactCenters API documentation for this.

Section~\ref{sec:mskxml} describes how XML is used by the simulator.
It provides a brief overview of XML as well as the main data
structures specific to our paramater file format.
We also give examples on how the HTML documentation for specific
parameters of the simulator can be retrieved.
The following subsections
give the available performance measures,
arrival processes, routing and dialing policies,
which are not listed in the HTML documentation for parameters.

% Types of performance measures, arrival processes,
% dialer's, and router's policies are represented as strings in the XML
% documents.
% These strings are not constrained by the XML
% schema representing parameter files, because plug-ins may
% eventually provide additional strings.
% As a result, the authorized values of these strings are not given in
% the HTML documentation of the schemas.
% The following sections therefore give the builtin
% types of performance measures,
% arrival processes,
% dialer's, and router's policies.

%Section~\ref{sec:mskccparams} details the elements and attributes used
%to describe a call center to be simulated.
%Section~\ref{sec:msksimparams} provides a detailed description of the
%parameters used to perform experiments.
%For each XML element described in these two sections, the
%its name,
%acceptable attributes, and allowed
%nested contents are specified and defined.  Required elements and
%attributes are identified.  If an element or attribute is not
%explicitly marked as required, it is optional and a default value or
%behavior is specified in the documentation.

Section~\ref{sec:mskexp} explains in more details the two supported
methods of experiment. In particular, it contains information on how
sequential sampling and batch means work in the simulator.

Section~\ref{sec:mskoutput} describes the output produced by the
simulator.  It describes the contents and format of any report
produced by the simulator, and how performance measures are regrouped
into matrices.  Every supported type of performance measure is also
presented in detail.
