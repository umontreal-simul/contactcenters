\section{Blend call center}
\label{sec:Blend}

This example is a simplified version of the model analyzed in
\cite{ccDES03a}, with reduced choice of distributions and dialing
policies.  Although it is a special case of the blend/multi-skill
generic simulator, it demonstrates how to use a dialer as well as
per-period probability distributions.
The simulated day starts at 8AM, ends at 2PM, and is
divided into three two-hours periods.
Calls received by the center are denoted \emph{inbound calls} and
arrive
following a Poisson process with randomized arrival rates
$B\lambda_p$.  To keep the
agents busy, a dialer is used to perform \emph{outbound calls} at some
times during the day.  Only a portion of the agents, denoted
\emph{blend}, are
qualified for handling these calls in addition to inbound ones.

Before 11AM, there is no outbound call, so all agents perform
inbound calls only.  After 11AM, a
portion of the agents becomes blend, and the dialer starts.
Inbound agents still serve inbound calls only while calls that cannot
be served immediately overflow to blend agents.
From 11AM to five minutes before the end of the day,
each time a service ends, the dialer tries to compose some phone
numbers.  The probability of a caller to be reached, or equivalently
of \emph{right party connect}, is specified for
each period.
Reached customers produce outbound calls which can only be served
by blend agents.  Outbound calls
then follow a non-Poisson arrival process which cannot be modeled in
simulation software such as Arena Contact Center Edition.

If no blend agent is
available when an outbound call succeeds,
the call is put in queue and considered as a
\emph{mismatch}.  The mismatched calls are put into a different
waiting queue
than the inbound calls in order to be served with the greatest
priority, because most reached callers will hang up without
even waiting.
Since in this model, there is no delay between the dialing and the
right party connect, mismatches only occur when the dialer tries to
call more customers than the number of free blend agents.

The program, shown in Listing~\ref{lst:Blend}, is very similar to the
other examples, except that a
dialer is used.

\lstinputlisting[
caption={A blend call center model},%
emph={create,main}, label=lst:Blend
]
{Blend.java}

We took most input values from table~2 in \cite{ccDES07a}, at
periods~15, 16, and~17.
From the input data, we get the expected number of arrivals per half hour.
We then divide these rates by 30 to get the arrival rates per minute.
The arrival process is constructed as in previous examples, using the
processed arrival rates.
The exponential patience time is set to
500~seconds, so we divide it by 60 to convert it in minutes.
The probability of balking also comes from the input data and is set
to $0.005$ for the three observed periods.
In the input data,
service times follow a gamma distribution with shape
parameter~$\alpha$ and scale parameter~$\beta=1/\lambda$.  Its mean,
expressed in seconds, is given by $\alpha\beta$.  To get the mean in
minutes, we divide $\beta$ by 60.  When we construct the gamma
generator, we will invert $\beta$ for the parameter to be compatible
with SSJ.
The gamma service times are generated using acceptance-rejection,
because inversion is too slow.

For the outbound type, we need the mean service time to be
$440.2$~seconds to follow the input data.  In contrast with the input
data in \cite{ccDES07a} setting the patience time for outbound calls
to~0, we fixed this patience time
to five
seconds for some outbound calls to get queued.  The probability of
reaching a caller and the staffing vectors were all taken from the
input data in \cite{ccDES07a}.

To take balking into account,
in the \texttt{new\-Instance} method of
\texttt{My\-Contact\-Factory}, a uniform is drawn.  If it
is smaller than the probability of immediate abandonment, the
patience time is 0.  Otherwise, it is exponential.  For the outbound
calls, the patience time is constant and set in the
\texttt{new\-Instance} method too.

The router is a simple queue priority router with fixed tables.
If an
inbound call arrives, it is served by an inbound agent or, if no
inbound agent is available, by a blend agent.  If an outbound call
arrives, it is served by a blend agent.  If an inbound agent becomes
free, it checks the inbound queue only.  If a blend agent is free, it
checks the outbound and the inbound queues, in that order.  When agents
arrive, they check for calls in the queues as when they become free.

The dialer is the newest and most complex part of this model.  To be
constructed, it requires at least a dialing policy, a random stream to
determine if the right party is reached, and a value generator giving
the probability of the right party to be reached.
Reached calls are notified to the router while
reached and failed calls are counted
using a \texttt{Contact\-Counter} which is a custom new-contact
listener.  Failed calls
include wrong party connects (answering machine or wrong person) as
well as busy signals or no answer.  The dialer
is connected to the router in order to be automatically started
(if it is enabled) each time any agent, whether inbound or blend, ends
a service.
It is also registered to be automatically initialized at the beginning
of each replication.

When the dialer starts, if it is enabled, it uses its policy to
determine how many calls to try.  Dialed calls are extracted from a
dialer list which is associated with the dialing policy.  In most
general settings, the list of outbound calls to try can depend on the
state of the system.  In this example, the dialer list is infinite,
i.e., it uses a contact factory to produce a call each time it is
asked to.  Retrials of failed calls are included in the calls extracted
from this infinite list.
For each tried call, a random number is drawn to
determine if it reaches the right party.  If it is successful, it is
sent to the router and the contact counter.  Otherwise, it is sent to
the contact counter only.

The dialer of this example uses a generic threshold-based policy which
works as follows.
It computes $\Ntf(t)$,
the total number of free agents in a \emph{test set} comprised in this
example of both agent groups.
It then checks that this
number is greater than or equal to a certain threshold fixed to~4 in
this example.
If the condition holds, the dialer then computes $\Ndf(t)$, the number
of free agents in a \emph{target set}.  In this example, this set
contains blend agents only.
If $\Ndf(t)$ is greater than or equal to a second threshold fixed to~1
in this example, the
dialer tries to reach $2*\Ndf(t)$ customers.

\texttt{in\-Queue\-Size} and \texttt{out\-Queue\-Size} are queue size
statistical collector being used for computing the time-average queue
size.  Each queue size collector monitors a single waiting queue and
computes the integrals for the queue size and optionally the number
of contacts of each type in the queue.  \texttt{inbound\-Volume} and
\texttt{blend\-Volume} are agent group volume statistical collector
used for the occupancy ratio.

Statistical collecting is performed the same way as in other
examples.  However, we want events to be counted for main periods
only.  As a result, events related to contacts arriving during the
preliminary period are ignored.  For agents' occupancy ratio and
time-average queue size, we need to create a custom period-change
listener.  At the end of the preliminary, the agent group volume and
queue size counters are reset to keep statistics about main periods
only.  At the beginning of the wrap-up period, the necessary integrals
are obtained before they are updated with observations from the
wrap-up period.
In addition, to get the time-average queue size over the opening
hours,
we divide the integral of queue size by the appropriate time
duration.  If we use the \texttt{average} method, the returned average
is on all simulation time, including the complete preliminary period.
Of course, our custom period-change listener needs to be registered
with the period-change event to receive information; this operation is
done in the constructor of \texttt{Blend}.
Note that we could also solve this problem with an agent-group
listener, as we did in section~\ref{sec:occcor}.

Listing~\ref{res:Blend} presents the statistical
results of the call center.

\lstinputlisting[caption={Results of the program \texttt{Blend}},%
label=res:Blend,language={}
]{Blend.res}
