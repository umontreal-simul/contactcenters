\section{Bilingual contact center}
\label{sec:Bilingual}

This example represents a contact center serving an English and
a Spanish population.  On each weekday, contacts of each type arrive
following a Poisson process with rate $100/$h from~8AM to~9AM, and
$50/$h from~9AM to~5PM.
7 English-speaking, 7 Spanish-speaking, and 4
bilingual agents are available to serve contacts.  The contact
center provides 40 phone lines to accommodate contacts.

When a contactor
arrives, the router randomly selects a free agent or queues the
contact if there is no free agent.
Patience times are i.i.d.\ exponential variables
with mean 2~minutes.  I.i.d.\ service times follow the uniform distribution,
ranging from 3~minutes to
7~minutes for English contacts, and from 4~minutes to 6~minutes for
Spanish contacts.

If abandonment occurs, with probability~$0.75$,
the contactor retries to join an agent after 20~minutes.
Arena Contact Center Edition supports a generalization of this concept
called \emph{contact backs}.  Whenever a contactor exits the system, he
has the possibility to enter back with a certain probability, after a
possibly random time.  In the call center terminology, contact backs
after abandonment are denoted \emph{retrials} whereas contact backs
after service are named \emph{returns}.  This differs from the concept
of returns, i.e., agents recontacting customers, implemented in Arena
Contact Center Edition.

The program estimates the service level for contact types~$k=0$
and~$1$,
and the service level over
all contact types, as well as the occupancy ratio for agent
groups~$i=0,1,2$ and the ratio over all agent groups.
Let $\SgK{k}(s)$ be the number of served contacts of
type~$k$ having waited in queue for less than $s$, and let $\AK{k}$
be the total number of arrivals of type~$k$.
The \emph{service level for type~$k$} is defined by
\begin{equation}
\frac{\E[\SgK{k}(s)]}{\E[\AK{k}]}.\label{eq:slarenak}
\end{equation}
The occupancy ratio in group~$i$ is defined by~(\ref{eq:occi}).
Since
\begin{eqnarray*}
S&=&\sum_{k=0}^{K-1} \SK{k},\\
A&=&\sum_{k=0}^{K-1} \AK{k},\\
N(t)&=&\sum_{i=0}^{I-1} N_i(t),\\
\Nb(t)&=&\sum_{i=0}^{I-1} \Nb[i](t),\\
&\mbox{etc.,}&
\end{eqnarray*}
equations~(\ref{eq:slarena}) and~(\ref{eq:occ}) can be used for
the overall service level and occupancy ratio.

The program given on Listing~\ref{lst:Bilingual} simulates the above
model and computes required statistics.

\lstinputlisting[
caption={A simulation program for a bilingual contact center},%
emph={create,main,MyRouter},label=lst:Bilingual
]
{Bilingual.java}

The program uses a helper class called
\texttt{Contact\-Center} providing convenience static methods for
initializing components of the contact center.

In real-life models, contact types and agent groups have names.
However, for efficiency, to distinguish contact types and agent
groups, the library uses numerical identifiers rather than strings.
On the other hand, replacing names with numerical identifiers reduces the
readability of the program and the produced statistical reports.
To make the program clearer, constants are associated
with each identifier.  The arrays \texttt{TYPENAMES} and
\texttt{GROUPNAMES} contain names for contact types and agent groups
being used for statistical reports.

In this model, we are interested in statistics for each contact type
separately as well as for all contact types.  In the program, many
scalars are then replaced by arrays of $K+1$ elements, and lists of
tallies are used instead of tallies for easier statistical collecting
and reporting.
Each list of tally has a \emph{global name} corresponding to a type of
performance measure, e.g., service level.  Each element in a list of
tallies also has its own
\emph{local name} corresponding to a particular measure, e.g., the service
level for a contact type, or the occupancy ratio of the agents in a a group.
The list of tallies are constructed using a \texttt{create} method
that also initializes the global and local names, and set
reporting options to show confidence intervals.

The model in Arena Contact Center Edition
is made of two parent groups which create two waiting
queues for contacts.  A \emph{parent group} is a set of agent groups
defining
a selection rule for the agents and assigning a preference for each
contact type.  To implement the parent group efficiently, a
queue is created for each contact type and the contact selection is
implemented in the router.

This contact center simulator requires a custom router to implement the
random selection of agents as well as the contact back mechanism.  This
router, represented by the \texttt{My\-Router} class,
extends \texttt{Single\-FIFO\-Queue\-Router} used in
section~\ref{sec:SimpleMSK}.   For this model, the type-to-group map
indicates that an English contact must be served by an English-speaking
or a bilingual agent, in that order.  A Spanish contact must be
served by a
Spanish-speaking or a bilingual agent.  The group-to-type map assigns
contact types to agents:  an English-speaking agent can serve English
people only whereas a Spanish-speaking agent can serve Spanish people.
The bilingual agents are generalists since they can serve any contact
type in this system.

The agent selection rule of the chosen router is not appropriate,
since we do not want specialists to have priority over
generalists in this particular example.
To implement random agent selection, the
\texttt{select\-Agent} method is thus overridden.
This method is called by the router when a contact arrives
and must be assigned an agent.  It selects an agent, starts the
service, and returns a reference to the end-service event.
The \texttt{select\-Uniform} helper method manages the random
selection automatically:  each agent group is assigned a probability
of selection given by the number of free agents in the group over the
total number of free agents capable of serving the contact.   If no
group contains free agents, a
\texttt{null} agent group reference is returned by
\texttt{select\-Uniform}, and \texttt{select\-Agent} returns
\texttt{null} to indicate the router that the contact must be queued.
Otherwise, the service of the contact is started.

When an agent becomes free, the
router tries to pull contacts from waiting queues.  In this example,
the contact selection rule of the inherited router does not need to be
customized.  For specialists, the
appropriate waiting queue is checked for a new contact which is removed and
served.  For generalists, both waiting queues are checked and the
contact with the longest waiting time is removed.

Further customization is needed to support retrials,
by overriding the \texttt{dequeued} method in
the router.  First, if the effective dequeue type is~0, the method returns
immediately because this is not an abandoning contact; the dequeue
type~0 is reserved for contacts being removed from the queue to be
served.  If the dequeue type is~1,
with some probability, a \texttt{Contact\-Back\-Event} is
scheduled after a fixed delay.  The contact then leaves the system,
using the
\texttt{exit\-Dequeued} method.  It is important to remove the contact from
the system in order to free its associated phone line.  When the
contact-back event occurs, the \texttt{new\-Contact} method is called
with the \texttt{Contact} object.
This is exactly what is performed by an
arrival process broadcasting a new contact to the registered
router.

However, some internal checks are performed
to avoid contacts entering multiple times in the system, which would
produce wrong simulation results.  When a contact exits the system, the
router sets an internal exit indicator
that must be cleared
before the contact can enter the router again.

To count the number of retrials, we need to define a subclass of
\texttt{Contact} with a new field being updated when retrial
occurs.  The subclass \texttt{My\-Contact} must be
referred to in \texttt{My\-Contact\-Factory} to get the appropriate
instances and typecasts are often needed in listeners such as
\texttt{My\-Contact\-Measures}.

% The \texttt{My\-Contact} subclass in the example defines a field named
% \texttt{serial\-Version\-UID}, because the class inherits the
% serializable property of \texttt{Contact}.  Although this is not
% necessary for this example, since contacts are not serialized here,
% the Java documentation recommends the association of a fixed serial
% version UID with each serializable class.
% Declaring this field can sometimes be useful to avoid warnings from
% the Java compiler.

As we can see in the program,
using arrays requires more work than scalars.
Of course, they must be constructed when declared or in the constructor of
the simulator.
Then, at the beginning of each replication, in the
\texttt{simulate\-One\-Day} method, each element must be initialized
separately.  Finally, in the exited-contact listener, events must be
counted in
two elements of the arrays.  When the event concerns a contact of
type~$k$, the index~$k$ of the arrays must be updated in addition to
index~$K$ for the measures for all types.
In addition, in \texttt{add\-Obs}, estimating the occupancy ratio
requires
$\int_0^T \Nb[i](t)\ dt$ and
$\int_0^T N_i(t)\ dt$ to be obtained for each agent group separately.

\texttt{report} is used to format
reports about each contact type or agent group, depending on the type
of performance measure.  The list report contains the
same information as the ordinary report, but, as shown on
Listing~\ref{res:Bilingual}, it is more condensed and
more appropriate for related performance measures.

\lstinputlisting[caption={Results of the program \texttt{Bilingual}},%
label=res:Bilingual,language={}
]{Bilingual.res}
