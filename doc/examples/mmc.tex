\section{$M/M/c$ queue and variants}
\label{sec:MMC}

\subsection{$M/M/c$ queue}

One of the simplest system that can be modeled with the
library is a
call center with a single call type, and a single agent group having a fixed
capacity~$c$, i.e., a $M/M/c$ queue with FIFO discipline.
This model is implemented for demonstration purposes only since
simple formulas are available to compute the average
waiting time, the average queue length, etc. \cite{pKLE75a}
It could also be implemented without ContactCenters, as shown in the
Examples user's guide of SSJ \cite{iLEC04j}.

Arrivals
follow an homogeneous Poisson process with rate
$\lambda$, and customers
are queued if they cannot be served immediately.  Abandonment is not
allowed and service times are i.i.d.\ exponential variables with mean~$1/\mu$.

The
system estimates
the service level, the expected speed of answer,
and the agents' occupancy ratio.
Let $\Sg(s)$ be the number of
served contacts
having waited in queue for less than an \emph{acceptable waiting time}
$s$, and let $S$
be the total number of served contacts.
Since no abandonment occurs, this is equivalent to
the number of arrivals.
 Let $\WS$ be the sum of waiting times for
served contacts.
The \emph{service level} is defined by
\begin{equation}
g_0(s)=\frac{\E[\Sg(s)]}{\E[S]}.\label{eq:slnoab}
\end{equation}
The \emph{speed of answer} is defined by
\begin{equation}
w_{\mathrm{X}}=\frac{\E[\WS]}{\E[S]}.\label{eq:sans}
\end{equation}
The \emph{agents' occupancy ratio} is defined by
\begin{equation}
o=\frac{\E\left[\int_0^{t_P} \Nb(t)\ dt\right]}{\E\left[\int_0^{t_P}
    (N(t)+\Ng(t)) dt\right]}.\label{eq:occ}
\end{equation}
$t_P$ corresponds to the time at which the contact center closes.
$\Ng(t)$ gives the number of \emph{ghost agents}, i.e.,
agents that must exit the system after finishing their current
service.  Since ghost agents appear only when $N(t)$ changes with time,
$\Ng(t)=0$ for this model.
To get an estimate of each performance measure of interest, we perform
$n$ independent replications of a simulation on a
finite horizon.

Most elements of the contact center simulator are
provided by the ContactCenters library as independent components.
The modeler's task consists of assembling these elements and
providing some glue code for statistical collecting, routing and stopping
conditions, some elements being optional.
For this example, whose code is presented on Listing~\ref{lst:MMC},
simpler programs can be written, especially
if the library is not used.  However,
we prefer to demonstrate how to balance simplicity and scalability in
a contact center simulation.

The first part of the program defines and constructs various objects
while the second part contains the simulation logic. The \texttt{main}
method, at the end of the file, constructs a $M/M/c$ queue simulator,
triggers a
simulation by calling \texttt{simulate}, and prints a statistical
report using \texttt{print\-Statistics}.  A \texttt{Chrono} is used to
determine the CPU time taken by the simulation.
The \texttt{simulate} method calls
\texttt{simulate\-One\-Day} $n$ times to perform the
replications while \texttt{simulate\-One\-Day} initializes the system,
performs the simulation, and collects some observations.

\lstinputlisting[
caption={A $M/M/c$ queue implemented with ContactCenters},%
emph={main,simulateOneDay,MyContactFactory,MyContactMeasures},
label=lst:MMC
]
{MMC.java}

At the beginning of the program, the necessary classes are retrieved
using \texttt{import} statements, and
a class named \texttt{MMC} and representing the simulator is declared.
Because variants of this example will be presented later as subclasses of
\texttt{MMC}, some indirections that might appear
unnecessary will be used in the program.

In this example and most of the following ones, hard-coded constants are
used as
input data for simplicity.  In general, it is strongly
recommended to use a parameter file, since it allows modification of
the parameters without recompiling the simulator.  The package
\path{umontreal.iro.lecuyer.xmlconfig} provides facilities to support
XML-based configuration files.  As with the underlying SSJ library,
the simulation
logic of ContactCenters makes no assumption about the time unit, but
input data should be defined consistently to avoid confusion in the
future.  In our examples, unless specified otherwise, the time
unit is the minute.  For example, the arrival rate specifies the
expected number of contacts per minute.

The objects representing components of the contact center, such as
the arrival process, waiting queue, agent group, as well as
the random variate generator for the service times can be accessed
through fields of the \texttt{MMC} class.
Since the components themselves do not compute any statistic, the
simulator defines some counters and statistical collectors to estimate
the performance measures of interest.
Counters are used during replications to compute observations while
statistical collectors are used at the end to collect them.
We use integers for counting events
rather than statistical probes to clearly show the difference between
the two steps during the simulation.
\texttt{vstat} is used to compute the integrals of the number of
agents over simulation time, for estimating the occupancy ratio.
Each agent group volume statistical object contains accumulates
observing the
total number of agents, the number of busy agents, etc.\ for a single
agent group.
For each counter, a statistical probe is defined to collect the
observations computed during each replication.

When instantiating the \texttt{MMC} class, the constructor, shown
after the declaration of fields, creates
the contact center objects and connects them together.
Inbound contacts are simulated using an arrival process which needs
to determine how to construct contacts.
Contacts are represented by \texttt{Contact} instances which are
constructed by a \emph{factory}, i.e.,
a mechanism permitting the instantiation of objects from the
\texttt{Contact} class, or any subclass.
In contrast, a constructor instantiates objects from a
single class.  A contact factory is defined by creating a class
implementing the \texttt{Contact\-Factory} interface.  The
\texttt{new\-Instance}
method, specified by this interface, must construct and return an
initialized instance of \texttt{Contact} (or any subclass of
\texttt{Contact}).  In this example, \texttt{My\-Contact\-Factory}
creates the contacts and sets their service times.

Arrival processes of different types could use the same contact
factory, but in this example, a single Poisson process is needed.
In addition to the contact factory, the process requires a
random stream for generating the exponential inter-arrival times, and
an arrival rate.

Constructing the agent group only requires the number of agents in
it.
This object manages the service of contacts and keeps counters for the
number of free and busy agents.  All agents of a group are usually
considered identical, simplifying the routing significantly.
The agent group automatically schedules events to manage service
termination if service times are associated with handled contacts.

The constructed standard waiting queue
orders waiting contacts based on their arrival times only,
implementing \emph{First In First Out} (FIFO) or \emph{Last In First
  Out} (LIFO) disciplines; the choice between FIFO and LIFO is made by
the routing policy.

The router uses a queue priority policy, but in
fact, any routing policy will act identically here since there is a
single call type and a single agent group.  The only important parameters
are the type-to-group and group-to-type maps which are always the same
for a single-type and single-group contact center.
To maximize the flexibility of the
library, no assumption is made about the connections between
components.  Arrival processes, agent groups,
and waiting queues must be manually connected
to the router.

The components of the library communicate using the \emph{observers} design pattern
\cite{iGAM98a}.  In that setting, an \emph{observable}
object, also called a \emph{broadcaster}, is
capable of transmitting some information to a list of registered
listeners known at runtime only.
A \emph{listener}, also called an \emph{observer}, is an object
receiving information
broadcast by an observable object.
In Java, listeners are required to implement a particular interface the
broadcaster uses to transmit the information through specified
methods.
Each component of the library defines its own listener interface to
avoid the necessity of type casting by observers.

Each time
a new contact is created by the factory and returned to the arrival
process, it is broadcast to a list of registered \emph{new-contact
  listeners}.  Such a listener can correspond to any object
implementing the \texttt{New\-Contact\-Listener} interface, which
specifies a \texttt{new\-Contact} method receiving \texttt{Contact}
objects.  For the contacts to be processed, the list bound to the
arrival process being used
must include a reference to the router, which can listen to new
contacts.

While it
does not interact with the arrival process, the router affects the
waiting queue by inserting new contacts when they cannot be served
immediately, or by
pulling ones when agents are free.  It also affects agent
groups by sending them contacts to serve.  The linking mechanism is
different from arrival processes
to emphasize the two types of relationships.

The simulator also needs to be notified
about contacts leaving the router through an
\emph{exited-contact listener}.  Such a listener object implements an
interface called \texttt{Exited\-Contact\-Listener} specifying methods
to receive exited contacts.  Each method transmits all available
information, permitting the simulator to count events under complex
conditions.

These
connections can easily be changed during the life cycle of the
simulator:  new listeners can be added, and old ones can be removed.
It is even possible, with some care, to completely replace the router
at runtime, allowing different routing policies to be tested without
recreating the entire system.

The \texttt{simulate} method initializes the statistical collectors
and performs $n$ independent replications by using
\texttt{simulate\-One\-Day}.  The latter method may be considered as
the heart of the program.  First, the simulation clock is initialized, and
the event list is cleared by \texttt{Sim.init}.  An event for the end
of the day is then scheduled, and
all the contact center's components are reset, avoiding
any side effect from previous replications.  This initialization
disables the arrival
process.  Then, after the event
counters are reset to 0, the arrival process is started, scheduling
the first contact.  The simulation can now be started using
\texttt{Sim.start}.  This instructs SSJ to start executing events,
until the event list is empty.

When the first arrival occurs, the Poisson arrival process calls the
\texttt{new\-Instance} method of \texttt{My\-Contact\-Factory} to get
a new  contact object.
This method simply constructs the appropriate object and generates a
service time which is stored in the
contact object to be retrieved later by the agent group.
After the new contact is returned, it is broadcast to the router, and a
new arrival is scheduled.  The user only needs to
provide the \texttt{new\-Instance} method, the rest of the logic being
implemented once in \texttt{Contact\-Arrival\-Process}, and inherited by
\texttt{Poisson\-Arrival\-Process}.

In its \texttt{new\-Contact} method,
the router sends the newly-constructed contact to an agent, or
adds it to the waiting queue if no agent is available.
The way the agent or the waiting queue are selected, called \emph{agent
  selection} and \emph{queue selection}, respectively, depends on the
algorithm implemented in the chosen subclass of \texttt{Router}.
In this example,
when a service starts, a free agent becomes busy, and an event is
scheduled to happen after the service time elapses.
By default, the service time is extracted from contacts, but this can
be customized in many ways if needed, as we will do in
section~\ref{sec:Bank}.
Since the default service time is infinite, it is important to set a
service time in the contact factory, or change the service time
generator of the agent group.
If the contact is queued, the contact is added, with extra
information, to an internal linked list.

When a service ends, the router is notified,
the contact exits the
system, and it is broadcast to the \texttt{served}
method of \texttt{My\-Contact\-Measures}.
This method receives all information about the served contact
through the \emph{end-service event} object.  In this example, after a
service is
counted, the waiting time of the contact must be obtained: the served
\texttt{Contact} is extracted from the end-service event, and the
total queue time is queried.  Note that this obtains the
\emph{cumulative} queue time, i.e., the time spent by the contact in
all waiting queues.  In simple systems, this is the same as the waiting
time in the last queue since the contact waits only once.
Other information such as the starting time of the service, and the
reason why the service ended can be extracted from the event.
The two other methods of \texttt{My\-Contact\-Measures} contain no
code, because no contact is blocked or abandons in this model.

When an agent became free due to the service termination, the
router tries to perform \emph{contact selection}, i.e., it
scans waiting queues to assign a waiting contact to the free agent.
Without contact selection, all queued contacts would wait forever.  In
this example, the router simply takes the first contact
in the waiting queue, implementing a FIFO discipline.  If the queue is
empty at pull time, the affected agent remains free until the
following arrival.

When the simulation clock reaches \texttt{DAYLENGTH}, the
\texttt{End\-Sim\-Event} is executed, which calls \texttt{end\-Sim} to
disable the arrival
process.  This cancels the last scheduled arrival, breaking the life
cycle of the process.
If the arrival process is never disabled, the simulation will run
forever, and results will never be printed.
\texttt{end\-Sim} also stores the integrals of $N(t)$ and $\Nb(t)$ for
the occupancy ratio to exclude the wrap-up period following the
closing time.  No more arrival occurs, but in-progress and queued
contacts are served before the end of the replication.  After all
these services are over, \texttt{Sim.start} ends and the
\texttt{simulate\-One\-Day} method continues its execution.
Note that in a further example, the \texttt{end\-Sim} method will be
overridden to change how the simulation is terminated.

It is important not to end the simulation abruptly using
\texttt{Sim.stop}, otherwise the system will end non-empty:
queues will contain contacts, and some
agents will still be serving customers.  If the simulation is stopped
abruptly, the program needs to take this non-empty state into account
during statistical collecting.

The \texttt{addObs} method is called to add observations to
statistical collectors.  In further examples, this will be overridden
to change how observations are collected.
The number of served contacts as well as the service level for the
replication are added into tallies.  A tally for functions of multiple means is used for the
service level to get a ratio of averages instead of an average of ratios.
This estimates the long-term service level rather than the performance
measure during a single day.
The integrals for $\Nb(t)$ and
$N(t)$ are added to tallies for estimating the occupancy ratio.
As with the service level, we
try to estimate a long-term measure.

After $n$ independent replications of this process, we have
statistical results for the estimated performance measures, including
sample averages, sample variances, and Student-$t$ confidence
intervals.  For the ratios, confidence intervals are computed using
the delta theorem \cite{tSER80a}.  The \texttt{print\-Statistics}
method is used to show a statistical report similar to Listing~\ref{res:MMC}.

\lstinputlisting[caption={Results of the program \texttt{MMC}},%
label=res:MMC,language={},float=htb
 ]{MMC.res}

\subsection{Supporting abandonment}

To support abandonment, some modifications to the previous program are
necessary: patience times must be generated and associated with
contacts, and the contacts having abandoned need to be counted.  Even
if the abandonment count is not required, the service level
computation must be
altered as follows to take this new aspect into account.

Let $\Lb(s)$ be the number of
contacts having abandoned after a waiting time greater than $s$, and
let $L$ be the total number of abandoned contacts.  The new service
level estimator is defined by
\begin{equation}
g_1(s)=\frac{\E[\Sg(s)]}{\E[S + \Lb(s)]}.\label{eq:slab}
\end{equation}
If abandonment is disabled, the new estimator reverts
to~(\ref{eq:slnoab}).

Listing~\ref{lst:MMCAb} presents an extension of the previous
program supporting abandonment.  Instead of rewriting the whole
simulator, we inherit from it and change its behavior by overriding
appropriate methods.

\lstinputlisting[
caption={An extension of the $M/M/c$ model supporting abandonment},%
emph={main,MyContactFactoryAb,MyContactMeasuresAb},
label=lst:MMCAb
]
{MMCAb.java}

Patience times are i.i.d.\ exponential variables with mean~$1/\nu$,
and are generated the same way as service times.
To support this aspect,
new fields containing the abandonment rate $\nu$ as well as a random
variate generator for the patience time are defined.
Then, a new counter and a new statistical
collector are declared for the number of abandoned contacts.
A statistical probe is also defined for the new service level estimator.
The logic of the main method is the same as for the previous example,
except it constructs an instance of \texttt{MMCAb} rather than
\texttt{MMC}.

The constructor first calls the superclass' constructor to create the
contact center, and alters the constructed system in the following
ways.  When contacts are created by the factory, they
must now be assigned a patience time.  To
achieve this result, the factory bound to the arrival process is
replaced by a new one called \texttt{My\-Contact\-Factory\-Ab}.  A
second exited-contact listener is also connected to the router in order
to count abandoned contacts.  The old listener, registered by the
superclass,
as well as the new one we have just created, will both be called by
the router when broadcasting an exited contact.  This way, is is not
necessary to completely rewrite the event-processing code.

The \texttt{simulate} method needs to be overridden to initialize the
added collectors.  \texttt{simulate\-One\-Day} is also overridden to reset
the new counter at the beginning of each replication.  All the code in
the superclass is reused, and overridden methods are called instead of
old ones.

\texttt{My\-Contact\-Factory\-Ab} extends the previous contact
factory to set the patience time of newly-constructed contacts. The
\texttt{new\-Instance} method of the superclass is first used to create the
contact, then a patience time is set in a way similar to the service time.

Abandonment is managed automatically by the waiting queue as follows.
Rather than contact instances, the queue stores objects representing
simulation events
happening at the time of automatic removal, e.g., abandonment,
disconnection, transfer to another queue, etc.
In the previous subsection, these \emph{dequeue events} were never
scheduled,
because patience times default to infinity if unspecified.  The
patience time, more generally the \emph{maximal queue time}, is by
default extracted from the contacts, the same way as the service
time.


\lstinputlisting[caption={Results of the program \texttt{MMCAb}},%
label=res:MMCAb,language={},float=htb
 ]{MMCAb.res}

The
\texttt{My\-Contact\-Measures\-Ab} class defines the \texttt{dequeued}
method to count abandoned contacts.  In a way similar to
\texttt{served} in \texttt{My\-Contact\-Measures}, this method takes
an object representing a
dequeue event to receive maximal
information.  An abandoned contact is counted, and if the waiting time is
greater than or equal to $s$, a contact having abandoned after the
acceptable waiting time is added.  In contrast with the contact
factory,
\texttt{My\-Contact\-Measures\-Ab} has no inheritance relationship
with its counterpart in \texttt{MMC}.  Both inner classes are
independent observers performing complementary tasks.

The \texttt{addObs} method is overridden to collect the number of
abandoned contacts and to compute service level differently.
The superclass' method is called and adds observations in the old and
inherited collectors.

Finally,
\texttt{print\-Statistics} is overridden to take the number of
abandoned contacts into account.  Listing~\ref{res:MMCAb} presents the
results of the extended program.
Abandonment increases the service level, because it reduces the size
of the waiting queue and allows customers to be served more quickly.
The occupancy ratio of agents is reduced, because some customers
served in the $M/M/c$ system abandon.


\subsection{Clearing the queue at the end of the day}

If the contact center was very busy, a big queue could build up
during the day.  This would result in agents working for a long time
after their shifts.  To make the model more realistic without
supporting abandonment,
when the contact center closes, queued contacts may be disconnected
instead of being served.
Agents still terminate their services, but the queued contacts are not
served.  To implement this, at the end of the simulation, the
\texttt{queue.clear} method
is called to remove all contacts.  This method accepts a
\emph{dequeue type} giving the reason why contacts are removed.
The dequeue type~0 is reserved for contacts to be served while dequeue
type~1 is used for abandoned contacts.  In this example, we clear the
queue with an indicator different from~1 to permit statistical
collectors to distinguish abandonment from disconnection.
Listing~\ref{lst:MMCClrQ} presents a second extension of the
$M/M/c$ queue, with support for disconnection.  This new aspect has an
impact on the original service level estimator.  We therefore define a
new collector for a corrected estimator including the number of
disconnected contacts in its denominator.

\lstinputlisting[
caption={An extension of the $M/M/c$ model supporting disconnections},%
emph={main,MyContactMeasuresClr},
label=lst:MMCClrQ
]
{MMCClrQ.java}

\lstinputlisting[caption={Results of the program \texttt{MMCClrQ}},%
label=res:MMCClrQ,language={},float=htb
 ]{MMCClrQ.res}

As it was made for abandonment
in the previous example, the base $M/M/c$ program is extended to
support disconnection.  This time, a new contact factory is not
needed.  As with the previous example, a second
observer is connected to the router to count additional events.
The exited-contact listener simply counts a
disconnection when a contact leaves the queue.  However, if abandonment
was additionally supported, it would be important to adapt the
exited-contact listener to distinguish abandoned and disconnected
contacts.  The \texttt{end\-Sim} method is
overridden to clear the waiting queue.

Listing~\ref{res:MMCClrQ} presents
statistical result of this program.
Clearing the waiting queue does not have a great impact on the
performance of the contact center, because the queue at
the end of the day is not too large.

\subsection{Limiting the queue capacity}
\label{sec:qclimit}

The capacity of a real system is never infinite.  For example, call
centers own a certain number of phone lines, and a caller arriving at
a time when all lines are busy cannot be served or wait in queue; it is
\emph{blocked} by the system.  To support this aspect,
the \texttt{Router} class defines the
\texttt{set\-Total\-Queue\-Capacity} method to limit the number of queued
contacts.
Blocked contacts are notified to the \texttt{blocked} method of registered
exited-contact listeners and can be counted too.  With this mechanism,
a limitation can be enforced on the
capacity over all waiting queues only.  In section~\ref{sec:Telethon},
we will
see a more general way of limiting the capacity of a contact center,
by using trunk groups.

Listing~\ref{lst:MMCBlocked} presents an extension of the $M/M/c$
basic model with limited queue capacity.  The program is similar to
examples in previous subsections: we extend the \texttt{MMC} class to
alter the contact center and add new statistical counters.  The
estimator of the service level is also modified to add the number of
blocked contacts in its denominator.
This time, the \texttt{blocked} method of the exited-contact listener
is defined and simply increments the counter for the number of blocked
contacts.  The \texttt{bType} indicator may be useful to get the
reason of blocking.

\lstinputlisting[
caption={An extension of the $M/M/c$ model with limited queue capacity},%
emph={main,MyContactMeasuresBlocked},
label=lst:MMCBlocked
]
{MMCBlocked.java}

Listing~\ref{res:MMCBlocked} displays the statistical results of the
new program.  Limiting the capacity has an effect similar to
abandonment by limiting the number of contacts to serve.

\lstinputlisting[caption={Results of the program \texttt{MMCBlocked}},%
label=res:MMCBlocked,language={},float=htb
 ]{MMCBlocked.res}

\subsection{Preparing for parallel simulations}
\label{sec:MMCSim}

As mentioned in the overview, most examples of this guide use the
\texttt{Sim} static class for simplicity.
But
if several instances of a simulator are needed, with one instance
assigned to one thread, each call to
static methods in the \texttt{Sim} class must be
replaced with a call to methods in an instance of \texttt{Simulator}.
The program in Listing~\ref{lst:MMCSim} is an adaptation of
Listing~\ref{lst:MMC} showing how to do this.

\lstinputlisting[
caption={An extension of the $M/M/c$ model using \texttt{Simulator}},%
emph={main},
label=lst:MMCSim
]
{MMCSim.java}

First, a field of type \texttt{Simulator} named \texttt{sim} is added
to the program.
This simulator encapsulates a simulation clock and event list specific
to a \texttt{MMCSim} instance rather than having a clock and a list
shared among all instances.
Calls to \texttt{Sim.init} and \texttt{Sim.start}
in the program are then replaced
with calls to \texttt{sim.init} and \texttt{sim.start}.

The simulator must be passed to most objects of ContactCenters.
This includes contacts, arrival processes, statistical collectors
for agent groups, and any user-defined event.
Consequently, we had to add a constructor to the
\texttt{End\-Sim\-Event} class in order to pass the simulator, and
change how \texttt{arriv\-Proc} and \texttt{vcalc} are constructed.
The \texttt{My\-Contact\-Factory} class is also modified to pass the
simulator to each new contact.
Agent groups and waiting queues do not need the simulator, because
they use the one assigned to the contacts they process.

If one forgets to give the simulator to an object that needs one, no
compile or run time errors occur by default.
The object in question  silently uses the default simulator, which
could result in unexecuted simulation events or other unpredictable
errors.
To help in preventing this, we would like
the program to crash as soon as
possible if the default simulator is used.
This can be done by replacing the default simulator with a dummy
unusable object whose each method throws an
\texttt{Unsupported\-Operation\-Exception}; we do this at the
beginning of the \texttt{main} method in the program.
If the program does not throw an exception, one can
then be relatively
confident that the default simulator is never used.
