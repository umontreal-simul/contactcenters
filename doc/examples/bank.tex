\section{Bank model}
\label{sec:Bank}

This example represents a bank model where each agent can process all
contact
types, but agents serve their specialty more efficiently.  The bank
can receive contacts for Savings, Checking, and Account Balance,
and employs 4 Checking specialists and
3 Savings specialists.  The Checking and Savings contacts are
preferred to be handled by specialists when possible, because service
times are multiplied by a $0.75$ reward factor.  For the Account
Balance contacts, the system must use the two agent groups in a
balanced way, so agent selection is random and uniform among free
agents.  For this model, patience times are uniform and
service times are exponential.  Arrivals follow a
non-homogeneous Poisson process whose rate depends on the period of the
day and the contact type.
Table~\ref{tab:Bank} presents the rest of the
input data for this model.

\begin{table}[htb]
  \caption{Input data for the Bank contact center}
  \label{tab:Bank}
  \centering
  \begin{tabular}{|l|rrr|}\hline
    &Savings&Checking&Account balance\\\hline
    Patience&$U(3\mathrm{min},5\mathrm{min})$&$U(3\mathrm{min},5\mathrm{min})$&$U(2.5\mathrm{min},3.5\mathrm{min})$\\
    Service&Exp($5\mathrm{min}$)&Exp($5\mathrm{min}$)&Exp($1\mathrm{min}$)\\
    Awt&$60$s&$90$s&$30$s\\
    Arrivals from 8AM to 9AM&$4/$h&$40/$h&$100/$h\\
    Hourly arrivals from 9AM to 5PM&$2/$h&$20/$h&$50/$h\\\hline
  \end{tabular}
\end{table}

The program estimates the service level as
well as the occupancy ratio, defined by~(\ref{eq:slarenak})
and~(\ref{eq:occi}), respectively.
The program also estimates per-contact type occupancy ratio defined by
\begin{equation}
o_{i,k}=\frac{E\left[\int_0^T \Nb[i, k](t)\
    dt\right]}{E\left[\int_0^T (N_i(t) + \Ng[i](t)) dt\right]}
\end{equation}
for $i=0,\ldots,I-1$,
and the global contact-type specific occupancy ratio
\[o_{I,k}=\frac{E\left[\int_0^T \sum_{i=0}^{I-1}\Nb[i, k](t)\
    dt\right]}{E\left[\int_0^T \sum_{i=1}^{I-1}(N_i(t) + \Ng[i](t)) dt\right]}.\]
which is
not estimated by Arena Contact Center Edition.

In the Arena Contact Center Edition model, idle and busy costs are
assigned to the two
agent groups.  The expected busy cost of agents in group~$i$
$C_{\mathrm{b},i}$ is defined by
\[C_{\mathrm{b},i}=c_{\mathrm{b},i}E\left[\int_0^T \Nb[i](t)\ dt\right],\]
where $c_{\mathrm{b},i}$ is the cost of one busy agent in group~$i$
per simulation time unit.
Similarly, the idle cost of agents in group~$i$
is defined by
\[C_{\mathrm{i},i}=c_{\mathrm{i}, i}E\left[\int_0^T \Ni[i](t)\
  dt\right].\]
The cost of Savings specialists is 12\$/hour while the cost of
Checking specialists is 7.50\$/hour, whether the agents are idle or
busy.
Of course, more general costing models could be implemented by a
ContactCenters model, e.g., time-varying idle and busy costs.

Listing~\ref{lst:Bank} gives the implementation of the above model to
estimate various performance measures for each contact type as well as
aggregate measures for all types.

\lstinputlisting[
caption={A simulation program for a bank model},%
emph={main,MyRouter},label=lst:Bank
]
{Bank.java}

The \texttt{Bank} class representing the simulator uses the
\texttt{Contact\-Center} utility class as in the previous example.
Names are also associated with each contact type, agent group,
period, and (contact type, agent group) pair,
for clearer statistical reports.

This example demonstrates an alternative way for performing
statistical collecting.
Instead of built-in types or arrays for counting events, we
use special \emph{measure matrices} adapted for storing observations
for multiple
performance measures and periods.  For example, \texttt{num\-Arriv} is
a $(K+1)\times (P+2)$ matrix whose rows correspond to contact types and
columns correspond to periods.  The used contact sum matrix must be
constructed with the number of contact types and periods, and provides
an \texttt{add} method accepting an observation for contact type~$k$
during period~$p$.  If $K=1$, the matrix contains a single row.
Otherwise, each addition to element $(k, p)$ is automatically repeated
in the last row, at element $(K, p)$, to update the sum over all
contact types.
In this model, since the lengths of the preliminary and wrap-up
periods are 0, the columns~0 and~$P+1$ of the matrix will always
contain 0's.  However, in general, these columns could contain
non-zero values.  Similar matrices are constructed for other
statistics such as the number of blocked, abandoned, and served
contacts.

For the occupancy ratio, as in the previous example, a statistical
collector is constructed for each agent group. An agent group
volume statistical collector is also
a measure matrix, with rows for the number of busy agents, the
number of working agents, etc.
Since it contains a single column for the integrals over all
simulation time, each matrix of agent group measures
is wrapped into an integral measure matrix to get the integrals
for multiple periods, by recording the values of integrals each time a
period ends.
For $\Nb[i](t)$ and $N_i(t)+\Ng[i](t)$, \emph{measure sets} regrouping
the values for each agent group during each period are constructed.
A measure set is a matrix of measures whose rows are copied from other
matrices.
For example, \texttt{svm} is a measure set containing
the service volumes: for $i=0,\ldots,I-1$ and $p=1,\ldots,P$,
element $(i, p)$ corresponds to $\int_{t_{p-1}}^{t_p} \Nb[i](t)\ dt$,
and element $(I, p)$ contains $\int_{t_{p-1}}^{t_p} \Nb(t)\ dt$.
Additionally, the program can compute
$\int_{t_{p-1}}^{t_p}\Nb[i, k](t)\ dt$, the
number of busy agents in group~$i$ serving contacts of type~$k$,
because the \texttt{vstat} instances were constructed with $K$.  To
regroup the values, in
\texttt{dsvm}, each row~$j=Ki+k$, for $i=0,\ldots,I-1$ and
$k=0,\ldots,K-1$, contains integrals for $\Nb[i, k](t)$.  For $i=I$,
$KI+k$ contains the integrals for $\Nb[i](t)$.
\texttt{dtvm} is similar to \texttt{tvm}, except that each row is
duplicated $K$ times to get the same number of rows as \texttt{dsvm}.

For each matrix of measures, a matrix of collectors is declared.  Each
matrix of collectors contains the same number of rows than its
corresponding matrix of measures, and $P+1$ columns.  The first $P$
columns correspond to main periods while the last column contains
time-aggregate values for all periods.  Optionally, these
aggregate values may include the preliminary and wrap-up periods, but
this is not used in this example.
Each matrix of probes has a global name corresponding to a performance
measure, row names corresponding to type or group names, and column
names corresponding to period names.

Matrices of measures and probes are created in the
constructor to be added to lists.  This allows them to be
automatically initialized by the \texttt{init\-Elements} method.

The \texttt{simulate} method initializes all probes in the list
\texttt{rep\-Probes} and calls \texttt{simulate\-One\-Day} $n$ times
to perform the replications.  Each replication initializes the
simulator, the contact center components, and the matrices of
measures.  The arrival processes are turned on, and the simulation can
start.

The \texttt{add\-Obs} method uses a convenience method from
\texttt{Contact\-Center} and \texttt{Rep\-Sim\-CC} (another utility
class)
to convert matrices of measures into Colt
matrices.  Let $M$ be a $R\times (P+2)$ matrix of measures.
For $r=0,\ldots,R-1$ and $p=1,\ldots,P$, element $m(r, p)$
of the matrix of measures is copied into element $(r, p - 1)$ of the
Colt matrix.  Element $(r, P)$ is obtained by
\[\sum_{p=1}^P m(r, p),\]
for each $r=0,\ldots,R-1$.
For computing the costs, another convenience method in
\texttt{Contact\-Center} is used.

When a contact is served by a specialist, the service time is multiplied
by $0.75$.  However, at the time its service time is generated, if the
contact is not served immediately, we do not know the group of its
serving agent.
This problem is addressed by changing the
agent groups' service time generator.  Similarly to waiting queues, a
value generator is associated with an end-communication indicator.
By default, service times are
extracted from contacts and the end-communication type is~0.  This
can be
changed to any \texttt{Value\-Generator} implementation by calling the
\texttt{set\-Contact\-Time\-Generator} method in
\texttt{Agent\-Group}.
The default contact time generator supports multipliers similar to the
talk time multipliers of Arena Contact Center Edition.  To activate
this feature, the program
constructs a new \texttt{Contact\-Service\-Time\-Generator} and
assigns it to the end-communication type~0.
For example, the Savings specialists will
have a $0.75$ service time multiplier for Savings contacts and $1.0$
multipliers for other types.
Note that this does not affect random number synchronization if it is
used, since the random service times are generated at the contact's
creation and multiplied by a constant afterward.

The agents' proficiency rewards preferred contacts-to-agents
associations, but it is the task of the router to enforce the
preference.  The router is a modified
\texttt{Single\-FIFO\-Queue\-Router} whose type-to-group map specifies
that
Savings contact types should be routed to Savings specialists first,
then to Checking specialists if no Savings specialist is
available.   For Checking contact type, this is the reversed order.
For Account Balance, any order may be used.
The inherited routing policy implements the expected selection rule
for the two first types, but it enforces an unwanted priority for
Account Balance.  We therefore need to override the
\texttt{select\-Agent} method to redefine the scheme for this
particular
contact type.  The new method simply selects a Checking or
Savings specialist randomly and uniformly among the available
agents, or reverts to the superclass' selection rule if the contact
type corresponds to Savings or Checking.

The contact selection rule of the single FIFO queue router is not
appropriate for our needs, because it enforces no priority at all on
waiting queues.  We therefore need to override the
\texttt{select\-Contact} method to add such priorities.
The new method, which is called when an agent becomes free,
first tries to pick a contact in the queue corresponding to the
preferred type.  The \texttt{remove\-First} method removes the first
contact in queue with dequeue type \texttt{DEQUEUETYPE\_BEGINSERVICE}
(which corresponds to 0), and returns a dequeue event
reference from which information can be
extracted.  Note that the argument to \texttt{remove\-First} is the
dequeue type, not the index of the contact in the queue to be removed.
% Otherwise, the method would have been named \texttt{remove}.
The
removed contact is extracted from the event and handled to the
\texttt{serve} method of the agent group having a free member.
If the prioritized queue is empty, the method falls back to the
superclass' implementation which takes the contact with the longest
waiting time.

Listing~\ref{res:Bank} gives the statistical results for the
simulated Bank model.  We can notice that the agents mainly serve
Checking contacts.  Also, if we sum up the per-contact type occupancy
ratios for an agent group, we obtain its global occupancy ratio.

\lstinputlisting[caption={Results of the program \texttt{Bank}},
%basicstyle=\ttfamily\footnotesize,
xleftmargin=-1.5cm,linewidth=18cm,breaklines,prebreak={\char92},
label=res:Bank]{Bank.res}
