\section{Telethon call center}
\label{sec:Telethon}

This example, adapted from Rockwell's Arena Contact Center
Edition~8.0,
deals with the organization of a one-week pledge drive
local public radio station.  In this model,
on each weekday, from~6AM to~10AM,
the 24 phone lines of the contact
center accept contacts from donors while donations are managed by 12
volunteers.
If a contact cannot be served immediately when it enters the system,
it is added to a FIFO queue.  The contact
leaves the queue if
\begin{enumerate}
\item its waiting time is greater than its exponential
patience time with mean 2~minutes (abandonment),
\item its waiting time reaches 2~minutes (the contact leaves a message
  and is disconnected),
\item the contact center closes (disconnection),
\item or a volunteer becomes idle and serves it.
\end{enumerate}
Service times are i.i.d.\ exponential variables with mean
10~minutes, and arrivals follow a Poisson process whose
rate is 50 contacts per hour.

Since the expected number of
donors is limited, the contact center must minimize busy signals and
abandonment.  Key performance measures are blocked and abandoned
contacts as well as the expected speed of answer, which
is the expected time a contact must wait in queue
before it is served.
The program also estimates the service level defined as
\begin{equation}
g_3(s)=\frac{\E[\Sg(s)]}{\E[A]},\label{eq:slarena}
\end{equation}
where $A$ is the number of arrivals.
It also estimates the occupancy ratio of agents defined by~(\ref{eq:occ}).

\subsection{Implementing the model}

\lstinputlisting[
caption={A simulation program implementing the Arena Telethon example},%
emph={create,main,simulateOneDay},label=lst:Telethon
]
{Telethon.java}

Listing~\ref{lst:Telethon} gives the
implementation of the simple non-stationary contact center described
above.
As with previous examples, fields are used for input data and objects,
and a constructor is defined to create the
necessary elements of the contact center.
We first construct the event that will mark the beginning of new
periods.
In this example and all other sample programs adapted from Arena
Contact Center Edition, the horizon is set to a full week, and
simulation time is thus divided into
$24*7$ 60-minutes main periods.
The preliminary and wrap-up periods defined by the
period-change event have length~0, but they must be counted
in the total
number of periods, so we need $24*7+2$ periods.

As usual, we then construct an arrival process using a contact
factory, a random stream, and other parameters.
In contrast with Arena Contact Center Edition only expecting the
number of arrivals for each main period, ContactCenters requires the
user to specify which arrival process to use along with its
parameters.
The normalization is enabled for the given rates to be divided by the
period duration.

In contrast with most previous examples, the capacity of the system is
limited.  As a result, some donors may receive a busy signal and be
blocked.
It could be possible to limit the number of phone
lines by setting the queue capacity to 12 as shown in
section~\ref{sec:qclimit}.
With this setting, when the 12~agents are busy
and 12~contacts are in queue, all the phone lines are busy, the system
is full, and any new contact is blocked until a service ends or a donor
leaves the queue.  However, in
general, because the number of agents in the system changes, the limit
on queue capacity needs to be updated at the beginning of each period.
It is also
possible to associate some contact types, e.g., premium contacts using a
specific phone bank as opposed to ordinary contacts accessing an independent pool
of phone lines.  Therefore, a more general framework for capacity
limitation may be needed.
The ContactCenters library supports the same notion of
trunk groups as Arena Contact Center Edition, but it is disabled by
default.  A \emph{trunk group} is a set of communication channels
(e.g., phone lines)
being allocated during the lifetime of contacts.  For this facility
to be enabled, each contact needs to be linked to a trunk group by
using the \texttt{set\-Trunk\-Group} method.  A
channel is then allocated when the contact reaches the router, and
freed
when it leaves the system.  If a contact arrives when all channels of
its trunk group are busy, it is blocked.
To implement this feature
in the Telethon example, the constructor creates a
\texttt{Trunk\-Group} instance which is associated with new contacts in
the \texttt{new\-Instance} method of the contact factory.

The same constructor as in the two preceding examples is used to
create the agent group and define the shift of the agents.
In Arena Contact Center Edition, an agent group schedule defines the
shifts for agents whereas agent groups only specify their capacity.
The library does not define such an agent group schedule; agent groups
can be created, as in section~\ref{sec:MMC},
with a fixed number of agents which can change at any
time, or, as in sections~\ref{sec:CallCC} and~\ref{sec:SimpleMSK},
with a per-period number of members.
In the example, the number of agents is constant, but it must be set
to 0 when the
contact center closes.  Therefore, an array of $P+2$ elements is
constructed and any index corresponding to an opening hour is filled
with 12, while the other values keep their default, 0.

We then construct a standard waiting queue to use a FIFO discipline.
In Arena Contact Center Edition, each agent group is associated with a
waiting queue.  In the
ContactCenters library, the waiting queues and agent groups are
independent from each other for maximal flexibility.  It is the
router's task to create the connection between these two elements.

The same router as in sections~\ref{sec:MMC} and~\ref{sec:CallCC} is
used since there
is a single contact type and a single agent group.  However, to simulate the
same model as with Arena Contact Center Edition, the queues have to be
cleared when all the
agents go off-duty.  The \texttt{Router} class fortunately provides
some facilities to perform this clearing automatically if it is
activated by the \texttt{set\-Clear\-Waiting\-Queues} method.
For more simplicity, we do not simulate the
30-seconds delay necessary for a donor to leave the message before
being disconnected.  This aspect would require the definition of a
custom router, which will be studied in the next subsection.

The \texttt{simulate} and \texttt{simulate\-One\-Day} methods are very
similar to their counterparts in previous examples. However, the
simulation logic, i.e., which events are executed during
\texttt{Sim.start}, differs slightly, because queued contacts can be
disconnected.

Automatic queue clearing is implemented by the router while the
disconnection after 2~minutes must be implemented
by setting a secondary \emph{queue
  time distribution} through the \texttt{set\-Maximal\-Queue\-Time\-Generator}
method of \texttt{Waiting\-Queue}.  By default, a waiting queue only
has a
primary maximal queue time generator associated with abandonment,
i.e., dequeue type~1, and the
generated value is the patience time extracted from contact objects.
Each queue time distribution is given by an implementation of
\texttt{Value\-Generator} which specifies a \texttt{next\-Double}
method taking an argument of class \texttt{Contact} and returning a
number.  To support message leaving,
the constructor of \texttt{Telethon}
therefore creates a constant value generator for a
single contact
type and assigns it to dequeue type~5.  This value was chosen
arbitrarily, because any dequeue type
different from~0 or~1 could be used.

When a contact enters the previously-configured queue, if the
extracted patience time is greater than~2, the
maximal queue time is set to~2, and the dequeue type is set to~5.  If this
maximal queue time is elapsed before a contact is served, the contact
is disconnected, and is asked to leave a message.
Otherwise, if the patience time is smaller than or equal to~2, the
maximal queue
time is the patience time and the
dequeue type is~1.  In this case, an abandonment occurs if the maximal
queue time is elapsed before service starts.

As in previous examples, we use an exited-contact listener to count
events.  However, when abandonment,
disconnection, and message leaving are considered simultaneously,
counting the dequeued contacts requires special care, because the
\texttt{dequeued}
method in \texttt{My\-Contact\-Measures} receives various kinds of
events.  The effective dequeue type must therefore be used to
differentiate the
three types of removals.  Dequeue type~1 is used for abandonment,
type~5 is used for message leaving while the router defines its own
dequeue type for disconnection when there is no agent.
The dequeue type~0 is reserved for dequeued contacts
to be served; these contacts are not notified to the \texttt{dequeued}
method of \texttt{My\-Contact\-Measures} since they have not left the contact
center yet.

Statistical collecting is performed the same way as in the
previous examples:  during the simulation, values
are counted and summed up.
The waiting times of served contacts are also considered to get the average
speed of answer.  Another similar measure which is computed by the example
in section~\ref{sec:CallCC} but not
by this
program is the average waiting time
which includes the served and abandoned contacts.  At the end of each
replication, \texttt{add\-Obs} is used to add the observations to the
appropriate statistical collectors.

We can reduce the variance on the service level and the occupancy
ratio by
replacing the denominators with the true, known, expectations.  The
expected daily number of donors is $4*50=200$, so $\E[A]=1000$ for the
whole week, which excludes the weekend in this model.  Since $N_0(t)$
is constant during all the opening hours,
the integral of $N_0(t)$ is constant, because we never have ghost
agents.

However, to get an accurate estimator of the occupancy ratio, we
should not count the integral of busy agents during periods where
$N_0(t)=0$.  We could achieve this using ContactCenters (see
section~\ref{sec:occcor}), but the
resulting occupancy ratio would not correspond to the output given by
Arena Contact Center Edition.

Results of the simulation, presented on Listing~\ref{res:Telethon},
seem different from Arena Contact
Center Edition, because different random seeds are used in the
ContactCenters simulator.  If the system is simulated with the same
number of
replications in both programs, the confidence intervals intersects, partly
showing that the results are not significantly different.
The simplification of the model to avoid a custom router does not seem
to bias the estimators significantly.

\lstinputlisting[caption={Results of the program \texttt{Telethon}},%
label=res:Telethon,language={}
]{Telethon.res}

\subsection{Modeling the message delay}

After a contact waits for
two minutes in the model of Arena Contact Center Edition, it leaves a
message for a duration of 30~seconds before the phone line is released.
This delay cannot be modeled by
adding $0.5$ minutes
to the maximal queue time since an agent must not be able to pick a
contact having waited more than two minutes while he is leaving its
message.
This aspect cannot be simulated unless the
router is customized such as in Listing~\ref{lst:TelethonMsg}.

The
\texttt{Queue\-Priority\-Router} must be extended
to override its \texttt{dequeued} method being called
every time a contact is removed from a waiting queue connected to the
router.  If the effective dequeue type is~0,
the event must be ignored since the contact is about to be served.
Otherwise, if the dequeue type is~5,
an event is scheduled to happen
after 30~seconds ($0.5$~minutes).  This exiting event must be defined
as an inner class in the custom router to call the protected
\texttt{exit\-Dequeued} method which releases the phone line and
broadcasts the exited contact.
If the dequeue type is~1, the \texttt{exit\-Dequeued} method must be
used for the phone line to be released at the time of abandonment.
Note that this technique could also be used to model more realistic
random message delays.

In this model, the delay is too short to have
any impact on the results over all time.  We may observe an effect on
some individual periods, or if the arrival rates were higher.

\lstinputlisting[
caption={Extension of \texttt{Telethon} for the 30 second delay},%
emph={main,MyRouter},label=lst:TelethonMsg
]
{TelethonMsg.java}

\subsection{Correcting the estimator for occupancy ratio}
\label{sec:occcor}

The
integral for $\Nb[0](t)$ needs to be computed for the opening hours
only to get an unbiased estimator for the agents' occupancy ratio.
The statistical collecting in \texttt{vstat}
must be disabled when $N_0(t)=0$ in order to achieve this.

In Listing~\ref{lst:TelethonOcc}, we define a new statistical
collector for the corrected estimator as well as a new group-volume
statistical counter.
In the constructor, we call the superclass' constructor and connect a
custom listener to the agent group before we create our new group
volume statistical collector.

The custom listener, implemented in
\texttt{My\-Agent\-Group\-Listener}, reacts when the agent group is
initialized, and the number of
volunteers changes, i.e., goes from~0 to~12 or from~12 to~0.
Setting the agent group of a group volume statistical counter
modifies the list of listeners of the agent group.
However, modifications to the list of listeners during the broadcast
of an event can lead to unpredictable results.  As a result,
the program needs to create a support event that configures the
group volume statistical collector.
Moreover, to
prevent inconsistencies, the group volume statistical collector needs
to be initialized after the simulation clock is reset, which cannot be
done without rewriting \texttt{simulate\-One\-Day} or modifying the
base class \texttt{Telethon}.  Fortunately, when an agent group is
initialized, it notifies every registered listener.  We therefore
react to this event by initializing the new group volume statistical
counter.

When the number of agents goes from~12 to~0 at the end of the day,
statistical collecting for \texttt{vstat\-Corr} needs to be disabled.
This can be achieved by breaking the association between the agent
group and the statistical counter.  The internal counters then reset
to~0 until the association is restored, at the beginning of the next
day.

\lstinputlisting[
caption={Extension of \texttt{Telethon} for corrected occupancy ratio estimator},%
emph={main,MyAgentGroupListener},label=lst:TelethonOcc
]
{TelethonOcc.java}

Listing~\ref{res:TelethonOcc} presents the output of the modified
program.  We see that the corrected occupancy ratio is significantly
different from the ratio obtained using Arena Contact Center Edition.

\lstinputlisting[caption={Results of the program \texttt{TelethonOcc}},%
label=res:TelethonOcc,language={}
]{TelethonOcc.res}
