\section{Teamwork contact center}
\label{sec:Teamwork}

This models a contact center with complex routing logic in which a
contact is processed by several agents.  This model contains a single
contact type but several agent groups.  Contacts arrive following
a non-homogeneous Poisson process with a mean arrival rate of $200/$h from
8AM to 9AM and $100/$h from 9AM to 5PM, and are served by several
agents sequentially.

When a contact enters the system, it is routed to one of the two
receptionists, or
queued if the receptionists are not available.  Service times are
i.i.d.\ variables following
the triangular distribution with minimum 0.5~minutes, mode 1~minute,
and maximum
5~minutes.  After the service is over,
with probability~$0.2$, a \emph{conference} is initiated with the
accounting department if the corresponding agent is free.
When a conference occurs, the contact still keeps the
receptionist busy and requests a communication with a secondary agent,
from the accounting agent group, if
available.  The service time with the accounting agent is uniform $2*U(0,
1)$~minutes.  The time spent with the receptionist is the sum of the
triangular
service time, and the uniform conference time if conferencing was
required
and succeeded.  After the communication is over, the receptionist
transfers the contact to the manager or to the
technical
support.  Before going idle or accepting a new contact, the agent
performs some after-contact work, e.g., updating a folder.  The
after-contact work has an
exponential duration with mean 1~minute, and starts after the contact
is transferred, not after it leaves the contact center.
After-contact work is also performed by the technical support and
manager with the same distribution, but it is not performed by
conferenced agents.

With probability~$0.05$, the contact is transferred to the manager by the
receptionist before after-contact work begins.  If the manager is
unavailable, the contact can be queued.  After being served by the
manager, the contact exits the system.  The service time with the
manager is generated from the same distribution as the service time
with the receptionist, but the generated value is multiplied by~3.

With probability~$0.95$, the receptionist transfers the contact it
has served to one
of the five technical support agents instead of the manager.
In contrast with the manger, if the contact cannot be served
immediately, it
disconnects without waiting in queue.  The service time with the technical
support is generated from the same triangular distribution as the
receptionist, but the obtained value is multiplied by~10.
After the service is done, with probability~$0.2$, a conference with
the
development agent is initiated. If the developer is available, the
conference time is uniform $5*U(0, 1)$.

%\subsection{Implementing the model}

\lstinputlisting[
caption={A simulation program for a teamwork model},%
emph={main,MyRouter},label=lst:Teamwork
]
{Teamwork.java}

Listing~\ref{lst:Teamwork} presents the complete source code for this
example.  In the program,
a class \texttt{Teamwork} is
created, and constants are declared for input data.  Most events are
counted using integers since the system supports a single contact
type.  The occupancy ratios of agents require arrays because the system
has five agent groups.

In previous examples, the different steps of a service were not
modeled explicitly.
An exponential or uniform service time was generated and
used, and the end of service was automatically scheduled by
ContactCenters.  In this example, the service of a contact is divided
in three
parts: the communication time with the
agent, the optional conference time with another agent, and the
after-contact
time.  The library supports after-contact time, but
conferencing must
be implemented by the user, because an infinity of
models can be imagined.

In ContactCenters, the service of a contact
is a two-pass process composed of the
communication with the agent (first phase), and the after-contact work
(second phase).
When the first phase is over, the contact is notified to
exited-contact listeners by the router for statistical purposes, but
the busy agent is not freed up until the second phase is over.
In previous examples, since there was no after-contact work, the two
steps were ending at the same time.

The \texttt{set\-Default\-Service\-Time} method of \texttt{Contact} we used in
previous examples sets the contact
time to the service time, and resets the after-contact time to 0.
The two durations can be changed separately by using the
\texttt{set\-Default\-Contact\-Time} and
\texttt{set\-Default\-After\-Contact\-Time} methods,
respectively.

In section~\ref{sec:occcor}, we used an agent-group listener to track
the number of agents in a group, in order to disable statistical
collecting when $N(t)=0$.  Because such a listener can be notified
about the end of a service, one could think about using this mechanism
to implement conferencing.
However, conferencing cannot be modeled with an agent-group listener
because
the listener is notified only after the phases of the
service are terminated.
It is not possible to conditionally extend the service duration as it
is needed in a conferencing model.
To model such complex services, automatic service termination must be
disabled.  By default, contact times are extracted from contact
objects and are infinite if unspecified.
As with the waiting queue which does not
schedule the dequeue event when the maximal queue time is infinite,
the end-service event is not scheduled if the obtained service time is
infinite.  The event can then be manually scheduled to simulate a
simple
service, or a wrapper event can be used to add logic before the
service termination.

Usually, a patience, a contact (or handle) and an after-contact times
are the
only random variables associated with a contact.  Here, additional
variables are required: the contact time for each agent group
(including conference times), the
after-contact times, as well as probabilistic
branching decision variables.  These variates can be generated when
needed, but to maximize random number synchronization, it is better to
generate all random variates at the creation of the contact object
since all the information is already available at this moment.
We do not know in advance the exact path of the contact, but we know
which random variates it might need during its lifecycle.
To support this extra information, a
subclass of \texttt{Contact} called \texttt{My\-Contact} is created.
The \texttt{My\-Contact\-Factory}
instantiates \texttt{My\-Contact} objects instead of \texttt{Contact}
objects as before.  Unfortunately, the library's classes manage
\texttt{Contact} instances which must be casted to \texttt{My\-Contact}
as needed.

For each agent group not used for conferencing, i.e., reception,
manager and technical support, an exponential after-contact time is
generated and stored in contacts.  A custom value generator is
assigned to each agent group in
order to extract the appropriate time from the contact objects.
The after-contact time generation works the same way as the service
time and maximal queue time generation.

The router is the most complex part of this contact center.  This time,
we define a completely custom router which extends the \texttt{Router}
abstract class.  The abstract router is constructed with the supported
number of contact types, agent groups, and waiting queues.  In this
example, there are two waiting queues: a first one for
receptionists, and a
second one for the manager.  There are five agent groups: Reception,
Accounting,
Manager, Technical support, and Development.
Every incoming contact is
sent to the Reception by \texttt{select\-Agent}.
If there is no free
receptionist, \texttt{select\-Agent} returns \texttt{null} and the contact
is queued by \texttt{select\-Waiting\-Queue}.

When the communication part of a service is terminated, the
appropriate
agent group notifies it to the router which calls its protected method
\texttt{end\-Contact}.  By default, this method simply calls
\texttt{exit\-Served} in order for the contact to leave the system after
service.  We can implement contact transfer by overriding this method
as follows.  The end-contact event is ignored for the Accounting and
Development agent
groups since the method will be called again to end the communication
with the associated Reception or Technical support agent groups. When
the communication between the contactor and the receptionist is
terminated, the transfer occurs: with probability~$0.05$, the contact is
sent to the manager.  If the contact center is closed at transfer time,
the contactor is disconnected by calling the \texttt{exit\-Blocked}
method.
The blocking-type indicator, given to \texttt{exit\-Blocked}, is set
to 5 in order to distinguish this disconnection from true
blocking because of no phone line.
If there is no free manager, the contact is put
in the queue for the manager, i.e., the second waiting queue.  Otherwise,
the second service of the contact begins.
If the contact is transferred to technical support,
the system applies the same algorithm as for the manager, except
that there is no waiting queue and a conference can occur.  If no
technical support agent is available, the service of
the contact ends at the receptionist (\texttt{exitBlocked} is called).
For any other agent group, the \texttt{end\-Contact} method
calls \texttt{exit\-Served} for the served contact to leave the system.

When the service of a
contact by an agent begins, whether it is started by agent or contact
selection, the protected \texttt{begin\-Service}
method is called.  We override the method in our custom router to
schedule the
end-service event appropriately, or to construct and schedule a
wrapper event for conferencing.  For the case of the receptionist,
with probability~$0.2$, a wrapper
\texttt{Conference\-Event} is scheduled to manage the service
termination.  The end-service event is kept inside the conference event
for future use.
With probability~$0.8$, the regular end-service event is scheduled and
no conference event is constructed.
Similar logic is used for the technical support, with a different
secondary agent group.  In the case of the manager, the regular
end-service event is always scheduled.
For Accounting and Development, nothing happens in that method,
because the service termination is managed by the conferencing event.

Conferencing is implemented using an auxiliary wrapper event
containing
three fields: \texttt{es} for the end-service event associated
with the main agent, \texttt{es\-Conf} for the end-service event
bound to the conferenced agent, and \texttt{target\-Group}, a
reference to the
conferenced agent group.  The end-service event objects are only used
as information containers instead of being scheduled as
simulation events.  The conference event happens at
most twice: one time to start the conference, and a second time to
terminate the communication with the main
agent as well as the conference.  The conference happens only if the
conferenced agent is free.

Automatic  waiting queue clearing must be redefined since there is no
default data structure in use.  The \texttt{check\-Waiting\-Queues}
method, which is called each time $N_i(t)=0$ for an agent group~$i$,
is responsible for this operation.  If it is called for a
Reception agent, the Reception waiting queue is cleared.  If it is called
for the Manager, the Manager waiting queue is cleared.  Otherwise, nothing
is done.  The waiting queue is cleared only if there is no available
agent in te corresponding group.

The average speed of answer and service level performance measures can
be computed several ways.  Each contact in the system can
sojourn in the reception's waiting queue, the
manager's waiting queue, or both queues.
Using the \texttt{get\-Total\-Queue\-Time} method
of the class \texttt{Contact} as in the previous examples gives
the cumulative waiting time in both queues.  We could also get the
queue time for each waiting queue.
Here, to keep the program as simple as possible, the speed of answer
is assumed to be the total waiting time in both queues for served
contacts.  The average handle time is
computed by taking the contact time for the last agent, which is
extracted from the end-service event given to the \texttt{served}
method of \texttt{My\-Contact\-Measures}.

There are two types of disconnections: when agents go off-duty or
when no technical support agent is free.  The first case is handled
in \texttt{dequeued} as usual.  The second case is handled in
\texttt{blocked} since the disconnection was performed using
\texttt{exit\-Blocked}, in the \texttt{end\-Contact} method of
\texttt{My\-Router}.  The blocking type is tested to distinguish this
event from a blocking due to no available phone line.

We can see from the statistical report on Listing~\ref{res:Teamwork}
that our results are different
from Arena Contact Center Edition~8.0's.  This is due to several
factors and possible bugs in Arena Contact Center Edition arising when
contacts are served by several agents sequentially.

First, there are several ways
to define the same performance measure and Rockwell's documentation
often does not clearly specify what is estimated when service is
performed by multiple agents.  By
experimenting with simplified models, it is sometimes but not always
possible to guess what computation is made.
Inconsistencies such as service levels greater than 100\%
 even arise in some extreme conditions.
Without accessing the definitions of the Contact Center Arena
templates, which cannot be done without the Professional edition, an
explanation for these inconsistencies cannot be found.

\lstinputlisting[caption={Results of the program \texttt{Teamwork}},
xleftmargin=-0.8cm,linewidth=17cm,breaklines,prebreak={\char92},
language={},label=res:Teamwork]{Teamwork.res}

\begin{comment}
\subsection{Different results from Arena Contact Center Edition}

We can see from the statistical report on Listing~\ref{res:Teamwork}
that our results are different
from Arena Contact Center Edition~8.0's.  This is due to several
factors and possible bugs in Arena Contact Center Edition arising when
contacts are served by several agents sequentially.

First, there are several ways
to define the same performance measure and Rockwell's documentation
often does not clearly specify what is estimated when service is
performed by multiple agents.  By
experimenting with simplified models, it is sometimes but not always
possible to guess what computation is made.
Inconsistencies such as service levels greater than 100\%
 even arise in some extreme conditions.
Without accessing the definitions of the Contact Center Arena
templates, which cannot be done without the Professional edition, an
explanation for these inconsistencies cannot be found.

Our first simplified model we will call $MM$
disables conferencing and sends every contact
to the manager after they are served by the receptionist.  This way,
the model becomes similar to preceding
examples, except that the service is performed by two agents
sequentially.
By simulating 10 replications in Arena Contact Center Edition, we
obtain a speed of
answer of $1.54\pm 0.03$.  If, as with the code of this example,
the average speed of answer is computed
using the cumulative queue time for contacts served by the manager
after being served by the receptionist only,
the Java program gives $3.30\pm 0.08$, which
is significantly higher than Arena Contact Center Edition.  If the
performance measure is computed by
summing the waiting times at the receptionist only for contacts served
by the manager, we obtain
$1.66\pm 0.05$, which is better but still significantly different from
Arena Contact Center Edition.
% We could try other variations such as including the waiting times for
% contacts having been served by the receptionist but having abandoned
% before being served at the manager; this gives a still different
% speed of answer of $1.68\pm 0.03$.

We now increase the talk time multiplier at
the manager, which increases the service time for this agent
group without affecting the receptionist.  During these
longer services, a queue at the manager should build up and the average
waiting time should increase.  If we increase the multiplier from 3 to
100, the average speed of answer effectively increases
to $1.703\pm 0.02$ when computed by Arena Contact Center
Edition.  As a result, the estimator takes the manager agent group into
account.

We then tried computing the number of services in each agent group
separately as follows.  If a contact abandons at the receptionist, its
waiting
time is not counted as it is usual for the speed of answer.  If a
contact is served by a receptionist but abandons before reaching the
manager, its waiting time at the receptionist as well as one service
are counted.  If a contact is served both by the receptionist and the
manager, the cumulative waiting time as well as two services are
counted.  The resulting speed of answer drops
at $0.317\pm 0.008$, which is still different from Arena Contact
Center Edition.
As a result, we cannot determine the exact formula being applied by
Arena Contact Center Edition to get the speed of answer when multiple
agents serve a contact sequentially.

With the service time, we have the same problem, with additional
inconsistencies from Arena Contact Center Edition.
In the simplified model $MM$,
if we set a constant service time in Arena Contact Center Edition, we
can suppose that the
handle time is given by the service time in the last agent group only,
not the sum of service times for all traversed agent groups.  If we
keep a random service time and set a 0 talk time multiplier for the
second agent group, we obtain a 0 handle time, which still verifies
our hypothesis.  However, problems appear if service times are random
for both agent groups.  The
expected triangular service time can be obtained by
\[\frac{0.5 + 1 + 5}{3}=2.1666.\]
For the manager, the expected service time should therefore be $6.5$
because
the service time multiplier is 3.  For the Java program, the expected
handle time falls into the confidence interval if all contacts go to
the manager.  However, it is
significantly smaller in Arena Contact Center Edition, i.e., $5.34\pm
0.03$.  As a result, the handle time estimator of Arena Contact Center
Edition suffers
from inconsistencies that could not be explained.

An additional test shows up inconsistencies with the service level.
If, in the original Teamwork model, all contacts are directed
to the technical support with conferencing still disabled, the
simplified model
has a single waiting queue since no customer can wait for a
technical support agent.
If we set the talk and after-contact times to 0, all services
become instantaneous.  No customer is waiting and the service level
should be 100\%.  However, Arena Contact Center Edition gives a
service level of
200\%, which is not what we expected.  The number of contacts meeting
the service level target reported by the program is $13888.40\pm
116.48$, which is twice the number of arrivals ($6944.2\pm 58.24$).
If we enable after-contact work back, the number of contacts in
target becomes $10572.20\pm 88.85$, which is still greater than the
offered load.
With non-zero service times, the number of contacts in target is still
greater than the number of handled contacts: the service level
estimator seems to include abandoned contacts.  However, if we include
abandoned contacts in the estimator, all preceding
Arena Contact Center examples return a significantly higher service
level.

Finally,
with the original model, Arena Contact Center Edition gives an average
of $664.40\pm 12.81$ handled and $6160.60\pm 47.50$ abandoned
contacts.  However, our Java example returns more handled and less
abandoned contacts.
In the Arena Contact Center Edition model, the receptionists serve
less contacts, which results in more abandonment.
It seems that the receptionist is unexpectedly kept busy for all the
sojourn time of the served contacts, even after they are transferred to
the technical support or manager, and the after-contact work is
completed.
\end{comment}
