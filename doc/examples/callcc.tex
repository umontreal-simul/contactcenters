\section{A simplified call center with multiple periods}
\label{sec:CallCC}

This example comes from the SSJ User's Guide \cite{iLEC04j} and models
a call center with a single call type, a single agent group, but
multiple periods.  
Each day, the center operates for $P$ hours.  The arrival rate of
calls and the number of agents answering them vary during the day, but
they are constant within each hour.  

The simulation time $T$ of a
replication is divided into $P+2$ periods. 
We assume that the contact center is opened during $P$ \emph{main
  periods}.  Main period~$p$, for $p=1,\ldots,P$, corresponds to time
interval $[t_{p-1}, t_p)$, where $0 \le t_0 < \cdots < t_P$.
Since the simulation may start before the contact center opens and
stop after it closes, two extra periods are defined.
During the \emph{preliminary period}
$[0, t_0)$, the center is closed and no service occurs, but
arrivals may start during this period for a queue to build up.
During \emph{wrap-up period} $[t_P, T]$,
no arrival occurs, but in-progress
services are terminated and queued contacts are processed.

Calls arrive following a Poisson process with piecewise-constant
randomized arrival rate $B\lambda_p$ during period~$p$, where
$\lambda_p$ is
constant and $B$ is a random variable having the gamma distribution with
parameters $(\alpha_0, \alpha_0)$.
Thus, $B$ has mean~1 and
variance~$1/\alpha_0$, and it represents the busyness of the day.
The day is more busy than usual when $B>1$ and less busy than usual
when $B<1$.

Calls that cannot be served immediately are put in a FIFO queue as in
the previous example, but callers can abandon.  The i.i.d.\ patience
times are generated as follows: with probability $\rho$, the patience
time is 0, i.e., the caller abandons if he cannot be served
immediately.  With probability $1-\rho$, the patience time is
exponential with mean $1/\nu$.  Service times are i.i.d.\ gamma random
variables with shape parameter~$\alpha$, scale parameter~$\beta$,
and mean~$\alpha/\beta$.

During main period~$p$, $N_p$ agents are available to serve calls.
If, at the end of period~$p$, the number of busy agents is
larger than $N_{p+1}$, ongoing calls are completed, but new
calls are not accepted until the number of busy agents is smaller than
$N_{p+1}$.  During the preliminary period, there is no agent whereas
for the wrap-up period, $N_{P+1}=N_P$.

We are interested in estimating the long-term expected waiting time
among all calls, the fraction of calls having waited less than $s$
seconds, and the fraction of calls having abandoned.
The service level estimated by this program is defined as
\begin{equation}
g_2(s)=\frac{\E[\Sg(s) + \Lg(s)]}{\E[A]},\label{eq:sl2}
\end{equation}
which differs from~(\ref{eq:slab}).
Let $\WL$ be the sum of waiting times for calls having abandoned.
If $W=\WS+\WL$ is the total waiting time of all calls and
$A=X+Y$ is the number of
arrivals, the estimated performance measures are $\E[W]/\E[A]$,
$\E[\Sg(s) + \Lg(s)]/\E[A]$, and $\E[L]/\E[A]$.

The ContactCenters program presented on
Listing~\ref{lst:CallCC} is simpler than the one presented in SSJ's
User's Guide, because many complex aspects are hidden in the library.

\lstinputlisting[
caption={A call center with multiple periods},%
emph={main,simulateOneDay,MyContactFactory,MyContactMeasures},
label=lst:CallCC
]
{CallCC.java}

To be compatible with the SSJ example, this program reads the
parameters from a file with the same format.
However, the used file format is sequential, thus inappropriate for
complex models: input errors cannot be detected, and the format cannot
be extended easily.
A properties file, or an hierarchical format such as XML, is
strongly recommended instead of this format.
Parameters are read by
\texttt{read\-Data}, which is the exact
copy of the method found in the SSJ example,  and
stored in fields used for constructing the system.
In this program, the main time unit is the second rather than the
minute.

When reading $\lambda_p$ and $N_p$, $\E[A]=\sum_{p=1}^P\lambda_p$ is
computed.  The arrival rate for the two extra periods is set to 0.
After the data is read, the service times generator \texttt{gen\-Serv}
as well as the busyness generator \texttt{bgen}
are constructed.

The \emph{period-change event} is then created
to divide the simulation horizon into $P+2$ periods.
This event
is used to notify \emph{period-change listeners} (some arrival
processes,
agent groups, custom listeners, etc.)
when a new period starts.  It also provides
methods to convert simulation times to period indices and get the
start and
end times of a specific period.  It can manage fixed-sized main periods as
well as variable-sized ones.
Note that as any other simulation event, the period-change event has
an associated instance of the \texttt{Simulator} class.
If a \texttt{Simulator} instance is created
as in section~\ref{sec:MMCSim},
it must therefore be passed
explicitly to the constructor when creating the period-change event.

The Poisson arrival process is constructed with a period-change event,
a contact factory, an array of $P+2$ arrival rates, and a random
stream.  This process automatically registers as a period-change
listener to update the arrival rate at the beginning of
periods.  The arrival rates entered in the input file provide the
number of calls for
each hour while the Poisson arrival process needs these rates to be
relative to one second.
Since the period duration corresponds to one hour, i.e.,
\texttt{HOUR} seconds, input data could also be considered to
provide the arrival rate per period.
We can then get the appropriate scaling by turning normalization on
for the arrival process. When normalization is
enabled, the arrival process automatically divides arrival rates
by the period duration before using them.  If it is turned off (the
default), arrival rates are not transformed.

The agent group is constructed with the period-change event and an
array of $P+2$ elements containing the number of agents for each
period.
As the arrival process, the agent group is automatically registered as
a period-change listener for the number of members to be updated at
the beginning of periods.

As in the previous section, the router is constructed to
associate the single agent group with the single waiting queue.  The
same connections as for the $M/M/c$ queue are needed between the
components of the system.

The \texttt{simulate} method initializes the statistical collectors
and calls \texttt{simulate\-One\-Day} $n$ times to perform
the replications.  The latter method initializes the simulation clock
and all counters.  It initializes the period-change event, which
resets the current period to 0.  The arrival process is initialized
with a new value of $B$, which is obtained using \texttt{bgen}.
Even though the arrival process is started before calling
\texttt{Sim.start}, thus at time~0, the first
arrival occurs after $t_0\ge 0$, because the arrival rate is 0 during
the preliminary period.  When the period-change event is started,
using \texttt{pce.start}, it is scheduled at time~$t_0$ to open the
call center.
The \texttt{Sim.start} method is then called, which starts
executing events.

After the simulation is finished, it is recommended to call the
\texttt{stop} method of \texttt{Period\-Change\-Event} for each
period-change
listener to be notified about the end of the simulation which is not
scheduled as a period-change event.
Observations are added to the appropriate tallies before
\texttt{simulate\-One\-Day} exits.

The contact factory is similar to the one in the previous section,
except that
the patience time is generated by \texttt{gener\-Patience}.  The
exited-contact listener contains code in the \texttt{dequeued} and
\texttt{served} methods to count arrived and abandoned contacts if
necessary.  In both cases, the waiting time of the contact is obtained
and added to a tally, and a good call is counted if the waiting time
is smaller than $s$.

The contact center closes at the beginning of the wrap-up period,
i.e., at time~$t_P$.  In
the $M/M/c$ queue, we had to manually disable the 
arrival process, because it was stationary.
In this example, because the arrival rate is set to 0 in the wrap-up
period, the arrival process is shut off automatically.  New arrivals
do not occur, but
all queued contacts are served before the simulation stops.
If $N_{P+1}$ was equal to 0 rather than $N_p$, agents would terminate
their ongoing service, but no new service would start during the
wrap-up period.  As a result, every contact still in queue during this
period would have to abandon.

After all replications are terminated,
the \texttt{print\-Statistics} method produces a report similar to
Listing~\ref{res:CallCC}. 

\lstinputlisting[caption={Results of the program \texttt{CallCC}},%
label=res:CallCC,language={},float=htb
 ]{CallCC.res}
