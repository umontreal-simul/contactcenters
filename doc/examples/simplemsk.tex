\section{A simple multi-skill contact center}
\label{sec:SimpleMSK}

This section presents an example of a contact center with three
contact types, two
agent groups, and three two-hours periods.  Contacts arrive
following a non-homogeneous Poisson process
with randomized arrival rate
$B\lambda_{k, p}$ for contact type~$k$ during period~$p$.
As in the previous section,
$B$ is a gamma busyness factor with parameters
$(\alpha_0, \alpha_0)$.

When a contact arrives, an agent is selected from a group depending on
the contact type.  Contacts of type~0 can only be served by agents in
group~0 while contacts of type~2 can only be served by agents in
group~1.  Contacts of type~1 are served by agents in group~0, or
in group~1 if no agent is free in group~0.
For contacts arriving during period~$p$,
service times are i.i.d.\ exponential random variables with
mean~$1/\mu_p$.
Agents in groups are not differentiated, and the total number of
agents can change from periods to periods.

A contact that cannot be served immediately is added to a waiting
queue corresponding to its type.  After a patience time, if service
has not started, the contact abandons.  For contacts arriving during
period~$p$,
patience times are i.i.d.\ exponential variables with mean~$1/\nu_p$.

We are interested in the overall service level defined
by~(\ref{eq:slab}), and the occupancy ratio
of the first agent group defined by
\begin{equation}
o_i=\frac{E\left[\int_0^{t_P} \Nb[i](t)\
    dt\right]}{E\left[\int_0^{t_P} (N_i(t)+\Ng[i](t))
    dt\right]}\label{eq:occi}
\end{equation}
where $\Nb[i](t)$ corresponds to the number of busy agents in
group~$i$, $N_i(t)$ corresponds to the total number of agents in group~$i$,
and $\Ng[i](t)$ corresponds to the number of ghost agents in
group~$i$, at time~$t$.
We also estimate $\E[\Sg[k, p](s)]$ for $k=0,\ldots,K-1$ and
$p=0,\ldots,P-1$, the number of contacts meeting the service level
target.

\subsection{Implementing the model}

Listing~\ref{lst:SimpleMSK} presents the code implementing this
model.

\lstinputlisting[
caption={A simple multi-skill contact center},%
emph={main,simulateOneDay,MyContactFactory,MyContactMeasures},
label=lst:SimpleMSK
]
{SimpleMSK.java}

As with previous examples, a class representing the simulator is
defined.  The \texttt{main} method constructs the simulator, runs a
simulation, and prints some statistics.

Fields are declared for contact center's components, statistical
counters, etc.
For the number of served and
abandoned contacts, simple integers are sufficient,
but a matrix is needed to get the number of contacts meeting service
level target, for each contact type and main period.

As in previous examples, a constructor is responsible for creating all
arrival processes, agent groups, waiting queues, etc.
The period-change event has a preliminary period of length~0,
two-hours main periods, and a wrap-up period with
random length.
For each contact type, a factory and an arrival process are
constructed.  In contrast with the previous examples, the
factory requires the contact type identifier to be given as an
argument.  In this example,
all the arrival processes use the same period-change event to be
notified when a new period starts.  Some arrival processes could of
course use different period-change events.
Constructing the agent groups only
requires the period-change event for $N_i(t)$ to be automatically updated,
and an array containing the number of agents for each period.

Service and patience times are generated using
random variate generators adapted for multiple periods.  Such
generators use a
period-change event to determine the current period and select a
period-specific generator to get random values.  The generic way to
construct them is to create a random variate generator for each
period and give the resulting array of generators, with a
period-change event, to the
constructor of \texttt{Multi\-Period\-Gen}.  For some distributions,
including exponential,
helper methods such as \texttt{create\-Exponential}
are available to construct the generators more
conveniently.

Because each waiting queue and agent group is
functionally identical, it is the task of the router to decide which contacts to
send to which agents or
queues, and from which queues contacts must be removed.
For the router to be constructed, a type-to-group map and a
group-to-type map are needed.  The selected
\texttt{Single\-FIFO\-Queue\-Router} class affects how these
structures are used.  Here, because we model a multi-skill contact
center, the data structures for the routing policy become important
and can greatly affect the performance of the contact center.
The choice of the subclass of \texttt{Router} affects how
contacts interact with the center.

The \texttt{simulate\-One\-Day} method must initialize several arrays
of elements, including arrival processes, agent groups, and counters.
The same value of $B$ is used for every arrival process, because the
busyness is not specific to a contact type.
As in the previous example, the period-change event and the
arrival processes are started before the simulation starts.

One factory object has been constructed for each arrival
process, the only difference between these objects being the value of
their \texttt{type} field.  This
avoids the necessity of one factory class for each contact type,
which greatly improves scalability.
The factory constructs a contact of the
appropriate type and generates a service and a patience times.

When a contact of type~0 arrives, the router takes the element~0 of
the type-to-group map, which corresponds to an ordered list containing
the
agent group~0 only.  Let $\Nf[i](t)$ be the number of idle agents in
group~$i$ available to serve contacts.
If $\Nf[0](t)>0$, the contact is served
immediately.  Otherwise, it is added to waiting queue~0.  Contacts of
type~2 are treated similarly.  For contacts of type~1, the router
obtains an ordered list containing~0 and~1.  If $\Nf[0](t)>0$, the contact
is served immediately by an agent in group~0.  Otherwise, it overflows
to the next agent
group in the list:  if $\Nf[1](t)>0$, the contact is served by an
agent in group~1.
Otherwise, it is added at the end of the queue~1.

When an agent within group~0 becomes free or is added, the router uses
the group-to-type map to obtain the ordered list $\{1, 0\}$.  The
chosen
router selects the queued contact with the longest waiting time rather
than using the order induced by the list.  If the waiting queues
accessible for agents in group~0 contain no contact, the agent
remains free until new arrivals occur.  Similar routing
happens for agents in group~1.  This is equivalent to managing a
single FIFO queue by merging all
per-type waiting queues, sorting contacts by increasing arrival
times, and removing the first contact the free agent can serve.

If \texttt{Queue\-Priority\-Router} was used as in previous examples,
the algorithm for agent selection would be the same, but waiting
queues would be scanned sequentially rather than considered as a
single FIFO queue.  For example, if an agent in
group~0 became free, the queue priority router would check for
contacts in queue~0 only when queue~1 is empty.

Each contact exiting the system is notified to the registered
exited-contact listener.  The \texttt{blocked} method does nothing
because
the capacity of the contact center is infinite; no contact is
blocked.   When a contact leaves the queue without service, it is
counted as having abandoned.
If its waiting time is greater than or equal
to $s$, a contact having abandoned after the acceptable waiting time
is also counted.  When a
contact is served, a new service is counted.  If its waiting time is
small enough, it is also counted as a contact meeting service level
target.

For a contact to be counted in \texttt{numGoodSLKP}, the main
period of its arrival must be
determined.  Arrivals occur in periods~$1,\ldots,P$,
corresponding to main periods~$0,\ldots,P-1$.
If the main period index, i.e., the period index minus one, is
negative or greater than or equal to~$P$,
the arrival occurred during the preliminary or
wrap-up periods, and the event is ignored.  When the arrival occurs
during a main period, the appropriate element of the matrix is
incremented.

\lstinputlisting[caption={Results of the program \texttt{SimpleMSK}},%
label=res:SimpleMSK,language={},float=htb,breaklines,prebreak={\char92}
 ]{SimpleMSK.res}

After the simulation stops, the \texttt{stop} method of
\texttt{Period\-Change\-Event} is called,
computed observations are added to collectors.
To keep the program simple, we estimate the occupancy ratio from
times~0 to~$T$ rather than~$t_P$.  Getting the correct ratio would
require the creation of a custom period-change listener to get
$\int_0^{t_P}\Nb[0](t)\ dt$ and $\int_0^{t_P} (N_0(t)+\Ng[0](t)) dt$ at
time~$t_P$, i.e., at the beginning of wrap-up period.
This will be done in a further example, in sections~\ref{sec:occcor}
and~\ref{sec:Blend}.
Listing~\ref{res:SimpleMSK} displays the results of the program when
performing 1000 independent runs.


\subsection{Adding a contact-by-contact trace}

For debugging or advanced statistical processing using tools such as
SAS or R, it may be
needed to get a contact-by-contact trace of the simulation. This can be done
easily by using observers, as shown in
Listing~\ref{lst:SimpleMSKWithTrace}.
This example is a simplified version of the trace facility available
in the generic simulator of call centers supporting blend and
multi-skill systems.
It shows how it is possible to extract information from a contact
object, and other parts of the simulator.

\lstinputlisting[
caption={Contact-by-contact trace added to the simple multi-skill contact
  center example},%
emph={main,simulateOneDay,ContactTrace},
label=lst:SimpleMSKWithTrace
]
{SimpleMSKWithTrace.java}

This program behaves the same as the program in the preceding subsection,
but it creates a text file named \texttt{contactTrace.log} with one line
for each processed contact, whether it has abandoned or was served.
One can then open the resulting (large) trace file with any text
editor, or make a program to parse it.
Listing~\ref{res:SimpleMSKWithTrace} displays ten lines of the trace
produced by the program.  We cannot display the full trace, because
its size is 2MB, even though we have simulated only five days.

The trace contains the following fields:
\begin{description}
\item[Step] The index of the replication the contact arrived in.
\item[Type] The type of the contact.
\item[Period] The period the contact arrived in.
\item[ArvTime] The simulation time of the contact's arrival.
\item[QueueTime] The waiting time in queue of the contact.
\item[Outcome] The outcome of the contact, can be \texttt{Served} or \texttt{Abandoned}.
\item[Group] The group index of the agent who has served the contact, or
  $-1$ for abandoned contacts.
\item[SrvTime] The service time of the contact, or \texttt{NaN} for
  unserved contacts.
\end{description}

\lstinputlisting[caption={Sample trace produced by the program \texttt{SimpleMSKWithTrace}},%
label=res:SimpleMSKWithTrace,language={},float=htb
 ]{SimpleMSKTrace.res}

The program defines a class named \texttt{SimpleMSKWithTrace} which
extends \texttt{SimpleMSK} to inherit the simulation logic of the
example in the previous subsection, but it adds a new exited-contact
listener to the router for logging contacts.
This trace manager, which is initialized at the beginning of the
simulation, writes a trace entry to a file for any exiting contact.
It does not have any effect on how the contacts are managed by the
simulator.

The new subclass defines three fields: the trace manager, the output
trace file, and the current
replication number.
The first two fields are initialized by the constructor, which also
attach the trace manager to the router, defined in the superclass
\texttt{Simple\-MSK}.
This way, every contact processed by the router is broadcast to the contact
trace manager.
Moreover, one could disable the contact-by-contact trace simply by unregistering
the trace manager from the router.

The last field is updated after each replication by an overridden
\texttt{simulate\-One\-Day} method, and is used when formatting trace
entries.

The overridden \texttt{simulate} method opens the trace file at
the beginning of the simulation, calls the \texttt{simulate} method
from the superclass to
perform the simulation, and closes the trace at the end.
We put the call to \texttt{close} into a
\texttt{finally} block that will be called even if an unexpected
exception is thrown by the simulator.
This prevents any loss of information caused by an unflushed buffer.

The main part of the program is the
\texttt{Trace\-Manager} inner class which is an exited-contact listener
producing the data written into the trace.
A trace manager has an associated print writer used to format trace
entries into the output file.

Log lines are constructed by the \texttt{dequeued} and \texttt{served}
methods of \texttt{Trace\-Manager}, using the \texttt{format} method
of \texttt{Print\-Writer}.
Note how the arrival time, the service time, the waiting time, the
contact type, etc.\ can all be extracted from the \texttt{Contact}
object.
The \texttt{format} method fills a user-provided pattern
by using the values given by the remaining arguments.  The print stream
takes the system's current locale and the line separator into account.

Arrival times and time durations are formatted with a fixed
number \texttt{timePrecision} of decimal digits of precision, which is
set to 3 in this program.
This allows for a better visual formatting of the data in the plain
text file.
One could easily use another format, e.g., XML, or use JDBC to
write data into a database instead of a file.

\subsection{Rerouting queued contacts}

Agents may sometimes be allowed to serve some types of contacts only
after the contacts have waited for some time in queue.
For example, we can modify the model of this example to allow agents
in group~1 to serve contacts of type~1 only after these contacts have
waited more than 12s.
This can be done by customizing the router as in
Listing~\ref{lst:SimpleMSKWithRerouting} for \emph{contact rerouting},
i.e., the router can, after a delay elapses, reprocess a queued
contact for a new agent selection with different criteria.
Note that the router in ContactCenters also supports \emph{agent
  rerouting} which is not covered in this example but
works in a way similar to contact rerouting:
if an agent stays free after the end of a service for a certain delay,
the router can reprocess the agent to do a new contact selection, with
different criteria.
Of course, agent rerouting requires the agent groups to keep track of
individual agents.

\lstinputlisting[
caption={Contact rerouting added to the simple multi-skill contact
  center example},%
emph={main,MyRouter},
label=lst:SimpleMSKWithRerouting
]
{SimpleMSKWithRerouting.java}

The customized router, named \texttt{My\-Router}, extends the
\texttt{Single\-FIFO\-Queue\-Router} used by the original example.
As a result, the agent and contact selections are performed the same
way as in the original example, but we use a different type-to-group
map which allows incoming contacts of type~1 to be served by agents in
group~0 only.

Contact rerouting works as follows.
When a new contact is notified to the router, an agent is selected by
the \texttt{select\-Agent} methods which is given the new contact as
its argument. This method must construct and return an end-service
event representing the served contact, or \texttt{null} if the contact
cannot be served.
When the contact cannot be served, it is added into a waiting queue by
the \texttt{select\-Waiting\-Queue} method of the router.
If the contact can be queued, the router obtains its initial
\emph{rerouting delay}, i.e., the time after which the queued contact
is reprocessed if still in queue.
If this delay is negative, no rerouting happens, which is the default.
When a positive rerouting delay elapses for a queued contact, the
router passes this queued contact to a second \texttt{select\-Agent}
method which accepts the dequeue event along with the number of
reroutings done. For the first rerouting, this corresponds to 0.
This new method returns an end-service event or \texttt{null}, exactly
as the ordinary \texttt{select\-Agent} method.
Of course, the scheme for agent selection implemented by this method
should differ from the scheme implemented by the ordinary
\texttt{select\-Agent} method.
If the contact still cannot be served, it remains in queue, and a new
rerouting delay is required.
The process continues until a negative delay is obtained, the contact
is served, or abandons.

The initial rerouting delay is obtained
using \texttt{get\-Rerouting\-Delay} with
$-1$ as the number of preceding reroutings. By
default, this method always returns $-1$, which results in no rerouting.
In our case, we return \texttt{DELAY} if the number of reroutings done
is $-1$, or $-1$ otherwise.
The \texttt{select\-Agent} method for rerouting applies the overflow
routing of the
original example, which allows contacts of type~1 to be routed to
agents in group~1 when no agent is available in group~0.

We also need to override contact selection in order to prevent an
agent in group~1 to dequeue a contact of type~1 if its waiting time is
smaller than \texttt{DELAY}.  We therefore override the
\texttt{select\-Contact}
method of \texttt{Router}, which accepts an agent group containing the
free agents, and returns the dequeue event representing the
contact to serve.
Our method scans the waiting queues accessible to the free agent, and
records the waiting queue whose first contact has the smallest enqueue
time. However, if the free agent is in group~1, the method filters out
queue~1 if the first contact has a waiting time smaller than
\texttt{DELAY}.
After the queue is selected, the first contact is removed, and
returned.  The contact is removed with dequeue type
\texttt{DEQUEUETYPE\_BEGINSERVICE} which is used to represent the
beginning of the service for a queued contact.

Listing~ \ref{res:SimpleMSKWithRerouting} shows that the modified
program produces slightly different results: the number of contacts
having abandoned is higher and the service level is smaller than with
the original model.  The simulator behaves as we would expect:
some contacts of type~0 have to wait 12s longer in queue before they
can be served, and some of them abandon while they were served in the
original setting.

\lstinputlisting[caption={Results of \texttt{SimpleMSKWithRerouting}},%
label=res:SimpleMSKWithRerouting,language={},float=htb,breaklines,prebreak={\char92}
 ]{SimpleMSKWithRerouting.res}
